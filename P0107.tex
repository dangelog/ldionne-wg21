\documentclass[11pt]{article}

\usepackage{xcolor}
\usepackage{fullpage}
\usepackage[colorlinks, allcolors=blue]{hyperref}
\usepackage{listings}
\usepackage{parskip}

\date{}
\title{Better support for {\tt constexpr} in {\tt std::array}}

\lstdefinelanguage{diff}{
  morecomment=[f][\color{blue}]{@@},           % group identifier
  morecomment=[f][\color{red}]{-},             % deleted lines
  morecomment=[f][\color{green!50!black}]{+},  % added lines
  morecomment=[f][\color{magenta}]{---},       % diff header lines
  morecomment=[f][\color{magenta}]{+++},
}

\lstdefinestyle{base}{
  language=c++,
  breaklines=true,
  basicstyle=\ttfamily\color{black},
  moredelim=**[is][\color{green!50!black}]{@}{@},
}

\newcommand{\added}[1]{\color{green}{#1}}

\lstset{
  basicstyle=\footnotesize\ttfamily,
}

\begin{document}

\maketitle\vspace{-2cm}

\begin{flushright}
  \begin{tabular}{ll}
  Document \#:&P0107R0\\
  Date:       &\date{2015-09-23}\\
  Project:    &Programming Language C++\\
              &Library Evolution Group\\
  Reply-to:   &\author{Louis Dionne}\\
              &\textless\href{mailto:ldionne.2@gmail.com}{ldionne.2@gmail.com}\textgreater
  \end{tabular}
\end{flushright}

\section{Introduction}
\cite{N3598} removed the implicit marking of {\tt constexpr} member functions as
{\tt const}. However, the member functions of {\tt std::array} were not revisited
after this change, leading to a surprising lack of support for {\tt constexpr} in
{\tt std::array}'s interface. This paper fixes this omission by adding {\tt constexpr}
to the member functions of {\tt std::array} that can support it with a minimal amount
of work.


\section{Motivation and Scope}
With the advent of generalized constant expressions, it can be useful to create
a {\tt std::array} inside a {\tt constexpr} function, and then modify it. Without
making some member functions {\tt constexpr}, this is impossible or overly difficult. For example, the following does not compile:

\begin{lstlisting}[style=base]
template <std::size_t N, std::size_t Size>
constexpr std::array<std::size_t, N * Size>
    cycle_indices(std::array<std::size_t, Size> a)
{
    std::array<std::size_t, N * Size> result{};
    for (std::size_t i = 0; i < N * Size; ++i) {
        result[i] = i % Size;
    }
    return result;
}
\end{lstlisting}

Adding {\tt constexpr} support for most of {\tt std::array}'s member functions
would be trivial and would make such code compile. However, this paper does not
propose systematically adding the {\tt constexpr} keyword to standard library
types that could support it, and it does not even add {\tt constexpr} to all of
{\tt std::array}'s member functions. Even though the author thinks that it should
eventually be done, the scope of this proposal is purposefully kept minimal.


\section{Impact on the Standard}
This proposal is a pure library extension. It does not require any new
language features, and it merely adds consistency to {\tt std::array}'s
interface.

\section{Proposed Wording}

Add to {\tt <array>} synopsis of \cite{N4296}:

\begin{lstlisting}[style=base]
template <class T, size_t N>
@constexpr@ bool operator==(const array<T,N>& x, const array<T,N>& y);

template <class T, size_t N>
@constexpr@ bool operator!=(const array<T,N>& x, const array<T,N>& y);

template <class T, size_t N>
@constexpr@ bool operator<(const array<T,N>& x, const array<T,N>& y);

template <class T, size_t N>
@constexpr@ bool operator>(const array<T,N>& x, const array<T,N>& y);

template <class T, size_t N>
@constexpr@ bool operator<=(const array<T,N>& x, const array<T,N>& y);

template <class T, size_t N>
@constexpr@ bool operator>=(const array<T,N>& x, const array<T,N>& y);
\end{lstlisting}

Add to \emph{23.3.2.1 class template array overview} of \cite{N4296}:

\begin{lstlisting}[style=base]
// iterators:
@constexpr@ iterator               begin() noexcept;
@constexpr@ const_iterator         begin() const noexcept;
@constexpr@ iterator               end() noexcept;
@constexpr@ const_iterator         end() const noexcept;
@constexpr@ const_iterator         cbegin() const noexcept;
@constexpr@ const_iterator         cend() const noexcept;

// element access:
@constexpr@ reference       operator[](size_type n);
@constexpr@ reference       at(size_type n);
@constexpr@ reference       front();
@constexpr@ reference       back();
@constexpr@ T*              data() noexcept;
@constexpr@ const T *       data() const noexcept;
\end{lstlisting}

Add to \emph{23.3.2.5 array::data} of \cite{N4296}:

\begin{lstlisting}[style=base]
@constexpr@ T* data() noexcept;
@constexpr@ const T* data() const noexcept;
\end{lstlisting}


\section{Discussion}
One might observe that some member and non-member functions were not made {\tt constexpr}
by this paper.

\begin{itemize}
    \item The {\tt rbegin}, {\tt rend}, {\tt crbegin}, and {\tt crend} member functions
          are not made {\tt constexpr}. The reason is that these functions return
          {\tt reverse\_iterator}s, which are not literal types. While we could have
          decided to go for it and make {\tt reverse\_iterator} a literal type, this is
          left to another proposal in order to leave this proposal small and uncontroversial.
          While leaving these functions non-{\tt constexpr} leaves some inconsistency in
          {\tt std::array}'s interface, this inconsistency is precedented by the overloads
          of {\tt rbegin}, {\tt rend}, {\tt crbegin}, and {\tt crend} for builtin array
          types, which are not {\tt constexpr} for the same reason.

    \item The {\tt fill} member function is not made {\tt constexpr} by this paper. The reason
          is that {\tt fill} can be implemented in terms of {\tt memset} for some types. Since
          {\tt memset} is not {\tt constexpr}, requiring {\tt fill} to be {\tt constexpr} would
          force it to be implemented using an explicit loop all the time. Such a pessimization
          is deemed unacceptable. Overcoming this limitation would most likely require the
          ability to overload on {\tt constexpr}, which is out of scope of this paper.

    \item The {\tt swap} member function and the overload of the {\tt swap} free function
          for {\tt std::array} is not made {\tt constexpr} by this paper. The reason is that
          the {\tt swap} function is not required to be {\tt constexpr} for other types,
          which means that {\tt std::array}'s {\tt swap} can't be {\tt constexpr} in the
          general case. To keep this proposal self-contained and minimal, this inconsistency
          could be tackled by a different paper adding general support for {\tt constexpr} in
          {\tt std::swap}. Another possibility would be to amend this paper and make
          {\tt swap} {\tt constexpr} for {\tt std::array} whenever it can be, i.e. whenever
          the elements of the array are {\tt constexpr} swappable.
\end{itemize}


\section{Implementation Experience}
This proposal was implemented and tested in libc++, and it seems to work just fine.


\section{Acknowledgements}
Thanks to Marshall Clow for providing comments, and to David Sankel for providing
comments and accepting to champion the paper in Kona.

\section{References}
\renewcommand{\section}[2]{}%
\begin{thebibliography}{9}

  \bibitem[N3598]{N3598}
    Richard Smith,
    \emph{constexpr member functions and implicit const}\newline
    \url{http://www.open-std.org/jtc1/sc22/wg21/docs/papers/2013/n3598.html}
  \bibitem[N4296]{N4296}
    Richard Smith,
    \emph{Working Draft, Standard for Programming Language C++}\newline
    \url{http://www.open-std.org/jtc1/sc22/wg21/docs/papers/2014/n4296.pdf}

\end{thebibliography}

\end{document}
