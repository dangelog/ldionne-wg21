\documentclass{wg21}

\usepackage{xcolor}
\usepackage{soul}
\usepackage{ulem}
\usepackage{fullpage}
\usepackage{parskip}
\usepackage{csquotes}
\usepackage{listings}
\usepackage{minted}
\usepackage{enumitem}

\lstdefinestyle{base}{
  language=c++,
  breaklines=false,
  basicstyle=\ttfamily\color{black},
  moredelim=**[is][\color{green!50!black}]{@}{@},
  escapeinside={(*@}{@*)}
}

\newcommand{\cc}[1]{\mintinline{c++}{#1}}
\newminted[cpp]{c++}{}


\title{String literals as non-type template parameters}
\docnumber{P0424R2}
\audience{Core Working Group}
\author{Louis Dionne}{ldionne.2@gmail.com}
\authortwo{Hana Dus\'{i}kov\'{a}}{hanicka@hanicka.net}

\begin{document}
\maketitle


\section{Revision history}
\begin{itemize}
  \item R0 -- Initial draft
  \item R1 -- Rewrite with different UDL form per EWG direction, and update motivation.
  \item R2 -- Incorporate feedback from EWG in Albuquerque:
              \begin{itemize}
                \item Use array syntax instead of pointer and length.
                \item Allow string literals as non-type template arguments.
                \item Initial wording attempt.
              \end{itemize}
\end{itemize}


\section{Abstract}
We propose allowing string literals as non-type template arguments. A string
literal would be passed as a reference to an array of characters:

\begin{cpp}
template <auto& str>
void foo();

foo<"hello">(); // creates a `constexpr char[6]` and passes a reference to it
\end{cpp}

To match this new functionality, we also propose adding a new form of the
user-defined literal operator for strings:

\begin{cpp}
template <auto& str>
auto operator"" _udl();

"hello"_udl; // equivalent to operator""_udl<"hello">()
\end{cpp}


\section{Motivation}
Compile-time strings are a sorely missed piece of functionality in C++.
Indeed, while we can pass a string as a function argument, there is no way
of getting a string as a compile-time entity from within a function. This
prevents a function from creating an object whose type depends on the
\emph{contents} of the string being passed. This paper proposes solving
this problem by allowing string literals as non-type template parameters.

There are many concrete use cases for this functionality, some of which were
covered in a previous version of this paper (\cite{P0424R0}). However, some
interesting use cases have recently come up, the most notable ones being
compile-time JSON parsing and compile-time regular expression parsing. For
example, a regular expression engine can be generated at compile-time as
follows (example taken from the \cite{CTRE} library):

\begin{cpp}
#include "pregexp.hpp"
using namespace sre;

auto regexp = "^(?:[abc]|xyz).+$"_pre;

int main(int argc, char** argv) {
  if (regexp.match(argv[1])) {
    std::cout << "match!" << std::endl;
    return EXIT_SUCCESS;
  } else {
    std::cout << "no match!" << std::endl;
    return EXIT_FAILURE;
  }
}
\end{cpp}

Under the hood, constexpr functions and metaprogramming are used to parse the
string literal and generate a type like the following from the string literal:

\begin{cpp}
RegExp<
  Begin,
  Select<Char<'a','b','c'>, String<'x','y','z'>>,
  Plus<Anything>,
  End
>
\end{cpp}

Since the regular expression parser is generated at compile-time, it can be
better optimized and the resulting code is much faster than \cc{std::regex}
(speedups of 3000x have been witnessed).

Similar functionality has traditionally been achieved by using expression
templates and template metaprogramming to build the representation of the
regular expression instead of simply parsing the string at compile-time.
For example, the same regular expression with \cite{Boost.Xpressive} looks
like this:

\begin{cpp}
auto regexp = bos >> ((set='a','b','c')|(as_xpr('x') >> 'y' >> 'z')) >> +_ >> eos;
\end{cpp}

It is worth noting that the specific use case of parsing regular expressions
at compile-time came up at CppCon during a lightning talk, and the room showed
a very strong interest in getting a standardized solution to this problem.
Today, we must rely on a non-standard extension provided by Clang and GCC,
which allows user-defined literal operators of the following form to be
considered for string literals:

\begin{cpp}
template <typename CharT, CharT ...s>
constexpr auto operator"" _udl();

"foo"_udl // calls operator""_udl<char, 'f', 'o', 'o'>()
\end{cpp}

With this proposal, we could instead write the following:

\begin{cpp}
auto regexp = sre::parse<"^(?:[abc]|xyz).+$">();
\end{cpp}

or, for those that prefer user-defined literals:

\begin{cpp}
using namespace sre;
auto regexp = "^(?:[abc]|xyz).+$"_pre;
\end{cpp}


\section{How would that work?}
The idea behind how this would work is that the compiler would generate a
constexpr array and pass a reference to that as a template argument:

\begin{cpp}
template <auto& str>
void f() {
  // str is a `char const (&)[7]`
}

f<"foobar">();

// should be roughly equivalent to

inline constexpr char __unnamed[] = "foobar";
f<__unnamed>();
\end{cpp}

Calling a function template with such a template-parameter-list works in
both \href{https://wandbox.org/permlink/zOOIb472ak9nBNMt}{Clang} and
\href{https://wandbox.org/permlink/8zpg3CLqzi9VTiuE}{GCC} today.


\section{Proposed wording}
Please note that this wording is a strawman aiming to convey the intent of
the authors. It was put together in a very short amount of time and changes
will be made to prepare it for review by CWG. The authors wanted to publish
a revision of this paper with the new design agreed upon by EWG.

This wording is based on the working draft \cite{N4700}. First, we allow
string literals as non-type template arguments. Secondly, we add the new
user-defined literal. We do this by splitting the term of art \textit{literal
operator template} into two terms, \textit{numeric literal operator template}
and \textit{string literal operator template}. The term \textit{literal
operator template} is retained and refers to either form. This is the approach
that was originally taken by Richard Smith in \cite{N3599}.

Remove in \textbf{[temp.arg.nontype] 17.3.2/2}:

\begin{quote}
  A \grammarterm{template-argument} for a non-type \grammarterm{template-parameter}
  shall be a converted constant expression of the type of the
  \grammarterm{template-parameter}. For a non-type \grammarterm{template-parameter}
  of reference or pointer type, the value of the constant expression shall not
  refer to (or for a pointer type, shall not be the address of):

  \begin{itemize}
  \item a subobject,
  \item a temporary object,
  \item \removed{a string literal,}
  \item the result of a \tcode{typeid} expression, or
  \item a predefined \tcode{__func__} variable.
  \end{itemize}
\end{quote}

Remove \textbf{[temp.arg.nontype] 17.3.2/4} (TODO: make red):

\begin{quote}
  \begin{note}
  A string literal
  is not an acceptable
  \grammarterm{template-argument}.
  \begin{example}

  \begin{codeblock}
  template<class T, const char* p> class X {
    // ...
  };

  X<int, "Studebaker"> x1;        // error: string literal as template-argument

  const char p[] = "Vivisectionist";
  X<int,p> x2;                    // OK
  \end{codeblock}
  \end{example}
  \end{note}
\end{quote}

Add after \textbf{[temp.arg.nontype] 17.3.2/2} (TODO: make green):

\begin{quote}
  When passed as a \textit{template-argument}, a string literal is a constant
  expression with external linkage.
  \begin{note}
    The intent is that \cc{f<"foobar">()} be equivalent to
    \begin{cpp}
      inline constexpr char __some_mangled_name_including_foobar[] = "foobar";
      f<__some_mangled_name_including_foobar>();
    \end{cpp}
  \end{note}
\end{quote}


Replace ``literal operator template'' with ``numeric literal operator template''
in \textbf{[lex.ext] (5.13.8)/3} and \textbf{[lex.ext] (5.13.8)/4}:

\begin{quote}
  [...] Otherwise, \cc{S} shall contain a raw literal operator or a
  \added{numeric} literal operator template (16.5.8), but not both. [...]
  Otherwise (\cc{S} contains a \added{numeric} literal operator template),
  \cc{L} is treated as a call of the form [...]
\end{quote}

Change in \textbf{[lex.ext] (5.13.8)/5}:

\begin{quote}
  If \textit{L} is a \textit{user-defined-string-literal}, let \textit{str} be
  the literal without its \textit{ud-suffix} and let \textit{len} be the number
  of code units in \textit{str} (i.e., its length excluding the terminating
  null character).
  \added{If \textit{S} contains a literal operator template with a non-type
  template parameter that can bind to \textit{str}, the literal \textit{L} is
  treated as a call of the form \texttt{operator "" X<str>()}. Otherwise, the}
  \removed{The} literal \textit{L} is treated as a call of the form
  \texttt{operator"" X(str, len)}.
\end{quote}

Change in \textbf{[over.literal] (16.5.8)/5}:

\begin{quote}
  \removed{The declaration of a literal operator template shall have an empty
  \textit{parameter-declaration-clause} and its \textit{template-parameter-list}
  shall have} \added{ A \textit{numeric literal operator template} is a literal
  operator template whose \textit{template-parameter-list} has} a single
  \textit{template-parameter} that is a non-type template parameter pack (17.6.3)
  with element type \texttt{char}. \added{A \textit{string literal operator template}
  is a literal operator template whose \textit{template-parameter-list} comprises
  a non-type \textit{template-parameter} \textit{str} that can bind to a string
  literal. The declaration of a literal operator template shall have an empty
  \textit{parameter-declaration-clause} and shall declare either a numeric
  literal operator template or a string literal operator template.}
\end{quote}


\section{Discussion on ODR}
We have two choices; either we don't mandate that equivalent string literals
share the same storage, or we do. If we do \textit{not} mandate that equivalent
string literals share the same storage, then template specializations with
equivalent string literals would potentially be different template
instantiations. This could lead to ODR issues:

\begin{cpp}
// in foo.hpp
template <auto& str>
struct foo { };

inline foo<"hello"> x;

// in a.cpp
#include "foo.hpp"

// in b.cpp
#include "foo.hpp"
\end{cpp}

In the above, there are two possibilities:
\begin{enumerate}
  \item \cc{x} is defined with a different type in \cc{a.cpp} and \cc{b.cpp}
        (ODR violation), or
  \item \cc{x} has the same type in both translation units (no ODR violation,
        only one copy of \cc{x} in the resulting program)
\end{enumerate}

We think this should \textit{not} be an ODR violation, and this is therefore
what the current wording achieves by giving external linkage to the string
literals used as \textit{template-argument}s. Under the covers, an
implementation would likely use the contents of the string literal to
produce the mangled name of the variable. Based on preliminary discussion
with implementers, this does not seem to be a problem.


\section{Potential generalization}
We could potentially make this applicable to arrays of arbitrary types,
with something like the following syntax:

\begin{cpp}
template <auto& array> void f();

f<{1, 2, 3}>(); // calls f with an array of type `int (&)[3]`
\end{cpp}

This is an interesting generalization, but the author prefers tackling
that as part of a separate proposal, since this proposal only targets
string literals and is very useful on its own.


\section{References}
\renewcommand{\section}[2]{}%
\begin{thebibliography}{9}

  \bibitem[P0424R0]{P0424R0}
    Louis Dionne,
    \emph{Reconsidering literal operator templates for strings}\newline
    \url{http://www.open-std.org/jtc1/sc22/wg21/docs/papers/2016/p0424r0.pdf}

  \bibitem[Boost.Xpressive]{Boost.Xpressive}
    Eric Niebler,
    \emph{Boost.Xpressive}\newline
    \url{http://www.boost.org/doc/libs/release/doc/html/xpressive.html}

  \bibitem[CTRE]{CTRE}
    Hana Dus\'{i}kov\'{a}
    \emph{Compile Time Regular Expression library}\newline
    \url{https://github.com/hanickadot/compile-time-regular-expressions}

  \bibitem[N4700]{N4700}
    Richard Smith,
    \emph{Working Draft, Standard for Programming Language C++}\newline
    \url{http://www.open-std.org/jtc1/sc22/wg21/docs/papers/2017/n4700.pdf}

  \bibitem[N3599]{N3599}
    Richard Smith,
    \emph{Literal operator templates for strings}\newline
    \url{http://open-std.org/JTC1/SC22/WG21/docs/papers/2013/n3599.html}

\end{thebibliography}

\end{document}
