\documentclass{wg21}

\usepackage{xcolor}
\usepackage{soul}
\usepackage{ulem}
\usepackage{fullpage}
\usepackage{parskip}
\usepackage{csquotes}
\usepackage{listings}
\usepackage{minted}
\usepackage{enumitem}

\lstdefinestyle{base}{
  language=c++,
  breaklines=false,
  basicstyle=\ttfamily\color{black},
  moredelim=**[is][\color{green!50!black}]{@}{@},
  escapeinside={(*@}{@*)}
}

\newcommand{\cc}[1]{\mintinline{c++}{#1}}
\newminted[cpp]{c++}{}


\title{Try-catch blocks in constexpr functions}
\docnumber{D1002R0}
\audience{Evolution Working Group}
\author{Louis Dionne}{ldionne.2@gmail.com}

\begin{document}
\maketitle


\section{Proposal}
Try-catch blocks can't currently appear in \cc{constexpr} functions:

\begin{cpp}
  constexpr int f(int x) {
    try { return x + 1; } // ERROR: can't appear in constexpr function
    catch (...) { return 0; }
  }
\end{cpp}

This paper proposes allowing this usage, but without changing the fact that
a \cc{throw} statement can't appear in a constant expression. This way,
compilation errors are still triggered by throwing in a \cc{constexpr}
function, and hence a \cc{catch} block is simply never entered. In other
words, try blocks are allowed in \cc{constexpr} functions, but they behave
like no-ops when the function is evaluated as a constant expression.

This proposal does not close the door to implementing error-handling in
\cc{constexpr} functions in the future if we so desire.

This proposal does not break any code, since \cc{constexpr} functions that
contain try-catch blocks are currently ill-formed.


\section{Motivation}
The underlying motivation is reflection and metaprogramming, just like
\cite{P0784R1}. Concretely, this limitation was encountered whilst surveying
\cc{std::vector} in libc++ with the purpose of making it \cc{constexpr}-enabled.
Indeed, \cc{vector::insert} uses a try-catch block to provide the strong
exception guarantee.


\section{Proposed wording}
This wording is based on the working draft \cite{N4727}. Change in
\textbf{[dcl.constexpr] 10.1.5/3}:

\begin{quote}
  The definition of a constexpr function shall satisfy the following requirements:
  \begin{itemize}
    \item[--] it shall not be virtual (13.3);
    \item[--] its return type shall be a literal type;
    \item[--] each of its parameter types shall be a literal type;
    \item[--] its \grammarterm{function-body} shall be \texttt{= delete},
              \texttt{= default}, or a \grammarterm{compound-statement} that
              does not contain
      \begin{itemize}
        \item[--] an \grammarterm{asm-definition},
        \item[--] a \texttt{goto} statement,
        \item[--] an identifier label (9.1),\added{ or}
        \item[--] \removed{a \grammarterm{try-block}, or}
        \item[--] a definition of a variable of non-literal type or of static
                  or thread storage duration or for which no initialization is
                  performed.
      \end{itemize}
  \end{itemize}
\end{quote}

Change in \textbf{[dcl.constexpr] 10.1.5/4}:

\begin{quote}
  The definition of a constexpr constructor shall satisfy the following requirements:
  \begin{itemize}
    \item[--] the class shall not have any virtual base classes;
    \item[--] each of the parameter types shall be a literal type\removed{;}\added{.}
    \item[--] \removed{its \grammarterm{function-body} shall not be a \grammarterm{function-try-block}.}
  \end{itemize}
\end{quote}


\section{References}
\renewcommand{\section}[2]{}%
\begin{thebibliography}{9}

  \bibitem[N4727]{N4727}
    Richard Smith,
    \emph{Working Draft, Standard for Programming Language C++}\newline
    \url{http://www.open-std.org/jtc1/sc22/wg21/docs/papers/2018/n4727.pdf}

  \bibitem[P0784R1]{P0784R1}
    Multiple authors,
    \emph{Standard containers and constexpr}\newline
    \url{http://www.open-std.org/jtc1/sc22/wg21/docs/papers/2018/p0784r1.html}

\end{thebibliography}

\end{document}
