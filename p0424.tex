\documentclass{wg21}

\usepackage{xcolor}
\usepackage{soul}
\usepackage{ulem}
\usepackage{fullpage}
\usepackage{parskip}
\usepackage{csquotes}
\usepackage{listings}
\usepackage[cache=false]{minted}
\usepackage{enumitem}

\lstdefinestyle{base}{
  language=c++,
  breaklines=false,
  basicstyle=\ttfamily\color{black},
  moredelim=**[is][\color{green!50!black}]{@}{@},
  escapeinside={(*@}{@*)}
}

\newcommand{\cc}[1]{\mintinline{c++}{#1}}
\newminted[cpp]{c++}{}


\title{Reconsidering literal operator templates for strings}
\docnumber{P0424R1}
\audience{Evolution Working Group}
\author{Louis Dionne}{ldionne.2@gmail.com}
\authortwo{Hana Dus\'{i}kov\'{a}}{hanicka@hanicka.net}

\begin{document}
\maketitle


\section{Revision history}
\begin{itemize}
  \item R0 -- Initial draft
  \item R1 -- Rewrite with different UDL form per EWG direction, and update motivation
\end{itemize}


\section{Introduction}
C++11 added the ability for users to define their own literal suffixes.
Several forms of literal operators are available, but none of them allows
getting a string literal as a compile-time entity from within the user-defined
literal operator. This prevents the user-defined literal to create an object
whose type depends on the \emph{contents} of the string literal. This paper
proposes solving that problem by allowing user-defined literal operators of
the following form to be considered for string literals:

\begin{cpp}
template <typename CharT, CharT const* str, std::size_t length>
auto operator"" _udl();

auto x = "abcd"_udl; // calls the above function
\end{cpp}


\section{History}
\begin{enumerate}
  \item \cite{N3599} (in 2013) proposed adding the missing literal operator
        using a \cc{char...} parameter pack, but the paper was rejected at
        that time with the following conclusion (\cite{CWG66}):
        \begin{quote}
          Revise with additional machinery for compile time string processing
        \end{quote}
  \item \cite{P0424R0} (the initial revision of this paper) was presented in
        Issaquah in 2016 and argued for adding the missing literal operator
        using a \cc{char...} parameter pack, but the paper was rejected
        because implementers were concerned with the compile-time cost of
        instantiating a function template with such a parameter pack.
  \item This paper addresses implementer's concerns by using a \cc{char const*}
        template parameter instead of a \cc{char...} parameter pack.
\end{enumerate}


\section{Motivation}
There are many use cases for such an operator, some of which were covered in
the previous version of this paper (\cite{P0424R0}). However, some interesting
use cases have recently come up, the most notable ones being compile-time
JSON parsing and compile-time regular expression parsing. For example, a
regular expression engine can be generated at compile-time as follows
(example taken from the \cite{CTRE} library):

\begin{cpp}
#include "pregexp.hpp"
using namespace sre;

auto regexp = "^(?:[abc]|xyz).+$"_pre;

int main(int argc, char** argv) {
  if (regexp.match(argv[1])) {
    std::cout << "match!" << std::endl;
    return EXIT_SUCCESS;
  } else {
    std::cout << "no match!" << std::endl;
    return EXIT_FAILURE;
  }
}
\end{cpp}

Under the hood, constexpr functions and metaprogramming are used to parse the
string literal and generate a type like the following from the string literal:

\begin{cpp}
RegExp<
  Begin,
  Select<Char<'a','b','c'>, String<'x','y','z'>>,
  Plus<Anything>,
  End
>
\end{cpp}

Since the regular expression parser is generated at compile-time, it can be
better optimized and the resulting code is much faster than \cc{std::regex}
(speedups of 3000x have been witnessed).

Similar functionality has traditionally been achieved by using expression
templates and template metaprogramming to build the representation of the
regular expression instead of simply parsing the string at compile-time.
For example, the same regular expression with \cite{Boost.Xpressive} looks
like this:

\begin{cpp}
auto regexp = bos >> ((set='a','b','c')|(as_xpr('x') >> 'y' >> 'z')) >> +_ >> eos;
\end{cpp}

It is worth noting that the specific use case of parsing regular expressions
at compile-time came up at CppCon during a lightning talk, and the room showed
a very strong interest in getting a standardized solution to this problem.
Today, we must rely on a non-standard extension provided by Clang and GCC,
which allows user-defined literal operators of the following form to be
considered for string literals:

\begin{cpp}
template <typename CharT, CharT ...s>
constexpr auto operator"" _udl();

"foo"_udl // calls operator""_udl<char, 'f', 'o', 'o'>()
\end{cpp}


\section{How would that work?}
The idea behind how this operator would work is that the compiler would
generate a constexpr string and pass that to the user-defined literal.
For example:

\begin{cpp}
template <typename CharT, CharT const* str, std::size_t length>
auto operator"" _udl();

"foobar"_udl;

// should be roughly equivalent to

constexpr char __unnamed[] = "foobar";
operator""_udl<char, __unnamed, sizeof(__unnamed)-1>();
\end{cpp}

Calling a function template with such a template-parameter-list works in
\href{https://wandbox.org/permlink/RBV6abYfNee94wlW}{Clang} and
\href{https://wandbox.org/permlink/rZEY8vDB5mHMPmmd}{GCC} today.


\section{Implementation experience}
As mentionned above, both Clang and GCC already provide a very similar
user-defined literal operator. The authors think that it should be fairly
straightforward to implement the operator proposed in this paper.


\section{Wording}
Wording will be provided if the proposal makes it through EWG.

\section{Discussion}
\begin{itemize}
  \item A reasonable person could argue that what we need is in fact a solution for
        compile-time strings instead of this operator. While we do need a solution
        for compile-time strings (and in fact one of the authors is involved in the
        compile-time programming proposals currently trying to solve that problem),
        the operator proposed in this paper solves a slightly different problem.
        Indeed, the problem we're solving here is the ability to access the contents
        of the string as constant expressions from within the body of the operator,
        which requires passing the contents of the string as a template parameter
        (since C++ does not have \cc{constexpr} parameters). We use a \cc{char const*}
        to pass it since that type is one we can already use for non-type template
        parameters. Another choice would have been to pass a \cc{std::string_view},
        but that would require expanding the set of types we can use as non-type
        template parameters, which is beyond the scope of what we want to do.

  \item A reasonable person could also request that more machinery for manipulating
        strings at compile-time be provided with this paper. Our answer is that such
        machinery is being worked on independently of this paper, and this proposal
        is useful on its own (even without such machinery). For information, the plan
        is to expand what parts of the standard library we can use at compile-time in
        such a way that we can use the same string processing machinery at compile-time
        and at runtime.
\end{itemize}


\section{References}
\renewcommand{\section}[2]{}%
\begin{thebibliography}{9}

  \bibitem[N3599]{N3599}
    Richard Smith,
    \emph{Literal operator templates for strings}\newline
    \url{http://open-std.org/JTC1/SC22/WG21/docs/papers/2013/n3599.html}

  \bibitem[P0424R0]{P0424R0}
    Louis Dionne,
    \emph{Reconsidering literal operator templates for strings}\newline
    \url{http://www.open-std.org/jtc1/sc22/wg21/docs/papers/2016/p0424r0.pdf}

  \bibitem[CWG66]{CWG66}
    Richard Smith,
    \emph{EWG Issue \#66}\newline
    \url{http://cplusplus.github.io/EWG/ewg-active.html#66}

  \bibitem[Boost.Xpressive]{Boost.Xpressive}
    Eric Niebler,
    \emph{Boost.Xpressive}\newline
    \url{http://www.boost.org/doc/libs/release/doc/html/xpressive.html}

  \bibitem[CTRE]{CTRE}
    Hana Dus\'{i}kov\'{a}
    \emph{Compile Time Regular Expression library}\newline
    \url{https://github.com/hanickadot/compile-time-regular-expressions}

\end{thebibliography}

\end{document}
