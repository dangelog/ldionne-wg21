\documentclass{wg21}

\usepackage[T1]{fontenc}
\usepackage[utf8]{inputenc}

\usepackage{xcolor}
\usepackage{soul}
\usepackage{ulem}
\usepackage{fullpage}
\usepackage{parskip}
\usepackage{csquotes}
\usepackage{listings}
\usepackage{minted}
\usepackage{enumitem}

\lstdefinestyle{base}{
  language=c++,
  breaklines=false,
  basicstyle=\ttfamily\color{black},
  moredelim=**[is][\color{green!50!black}]{@}{@},
  escapeinside={(*@}{@*)}
}

\newcommand{\cc}[1]{\mintinline{c++}{#1}}
\newminted[cpp]{c++}{}


\title{Controlling the instantiation of vtables and RTTI}
\docnumber{P1263R0}
\audience{EWG}
\author{Louis Dionne}{ldionne@apple.com}

\begin{document}
\maketitle

\section{Abstract}
It is currently not possible to control where the RTTI and the vtable of a type
are instantiated. It is also impossible to apply attributes to the vtable and
the RTTI of a type. This leads to difficulties when trying to control the ABI
of a shared library: one must apply attributes to and export the whole class
from the shared library, which is not an option some members of the class
should not be exported.

This proposal suggests providing explicit control over where the vtable and
the RTTI of a type are instantiated. By introducing syntax that allows referring
to those objects, it also becomes possible to apply attributes to them:

\begin{cpp}
template <typename T>
struct Foo { virtual ~Foo() { } };

struct Bar { virtual ~Bar() { } };

// Explicit instantiation declaration of RTTI/vtable
extern template Foo<int>::typeid;
extern template Foo<int>::virtual;
extern Bar::typeid;
extern Bar::virtual;

// Explicit instantiation of RTTI/vtable
template Foo<int>::typeid;
template Foo<int>::virtual;
Bar::typeid;
Bar::virtual;
\end{cpp}


\section{Motivation}
As outlined in the abstract, this proposal is relevant for libraries and
applications wishing to control the symbols exported in their ABI. In this
section, we outline a couple of motivations related to ABI control.

\begin{enumerate}
  \item \textbf{Controlling whether the vtable/RTTI is explicitly instantiated}\\
  Consider the following situation where we want to instantiate some member
  functions and the vtable of (a specialization of) a class template in a
  shared library:

  \begin{cpp}
  // in a header
  template <class CharT, class Traits = std::char_traits<CharT>>
  class basic_ios {
  public:
    bool operator!() const { /* ... */ }
    basic_ios& copyfmt(basic_ios const&) { /* ... */ }
    virtual ~basic_ios() { }
    // ...
  };

  extern template class basic_ios<char>;

  // in the shared library
  template class basic_ios<char>;
  \end{cpp}

  Here, we end up explicitly instantiating the RTTI, the vtable, and all the
  member functions of \cc{basic_ios<char>} in the shared library. However, if
  we don't want \cc{basic_ios<char>::operator!} to be instantiated in the
  shared library, we might be tempted to write the following (note that we
  don't externally instantiate \cc{basic_ios<char>::operator!}):

  \begin{cpp}
  // in a header
  template <class CharT, class Traits = std::char_traits<CharT>>
  class basic_ios {
  public:
    bool operator!() const { /* ... */ }
    basic_ios& copyfmt(basic_ios const&) { /* ... */ }
    virtual ~basic_ios() { }
    // ...
  };

  extern template basic_ios<char>& basic_ios<char>::copyfmt(basic_ios<char> const&);
  extern template basic_ios<char>::~basic_ios();

  // in the shared library
  template basic_ios<char>& basic_ios<char>::copyfmt(basic_ios<char> const&);
  template basic_ios<char>::~basic_ios();
  \end{cpp}

  Unfortunately, this solution does not work because the RTTI and the vtable are
  not instantiated in the shared library anymore. We are not aware of any way to
  solve this problem in the language.

  \item \textbf{Applying attributes to the vtable/RTTI}\\
  Another related problem is the application of attributes to the vtable and the
  RTTI of a class. Since there is no way to syntactically refer to those entities,
  it is impossible to apply attributes to them. Unfortunately, controlling the
  ABI of a shared library requires applying visibility attributes to those
  entities -- right now one can only apply those attributes to the whole class.
  For example, we'd like to write the following:

  \begin{cpp}
  // in a header
  #define HIDDEN __attribute__((__visibility__("hidden")))
  #define VISIBLE __attribute__((__visibility__("default")))

  namespace std HIDDEN { // hide everything from the ABI by default
    template <class CharT, class Traits = std::char_traits<CharT>>
    class basic_ios {
    public:
      bool operator!() const { /* ... */ }
      basic_ios& copyfmt(basic_ios const&) { /* ... */ }
      virtual ~basic_ios() { }
      // ...
    };

    // cherry-pick symbols to export from the shared library
    extern template VISIBLE basic_ios<char>::virtual;
    extern template VISIBLE basic_ios<char>::typeid;
    extern template VISIBLE basic_ios<char>&
                            basic_ios<char>::copyfmt(basic_ios<char> const&);
    extern template VISIBLE basic_ios<char>::~basic_ios();
  }

  // in the shared library
  template std::basic_ios<char>::virtual;
  template std::basic_ios<char>::typeid;
  template std::basic_ios<char>&
           std::basic_ios<char>::copyfmt(std::basic_ios<char> const&);
  template std::basic_ios<char>::~basic_ios();
  \end{cpp}

  \item \textbf{Controlling which translation unit the vtable/RTTI are defined in}\\
  Another benefit of this proposal would be to eliminate the need for a common
  technique called "anchor functions". This technique is used to control the
  translation unit in which the vtable and the RTTI of a type are instantiated.
  The technique consists in declaring (but not defining) a non-inline virtual
  function in the class, and defining that function in exactly one translation
  unit:

  \begin{cpp}
  // in header
  struct Foo {
    virtual void a() { }
    virtual void b() { }
    virtual void anchor();
  };

  // in exactly one TU
  void Foo::anchor() { }
  \end{cpp}

  This results in the vtable and RTTI being instantiated in the translation unit
  where the anchor function is defined. This technique relies on the fact that
  the vtable and RTTI are always defined in the translation unit where the first
  virtual function of the vtable is defined. Since the class contains a non-inline
  virtual function, it knows that this virtual function must be defined in
  exactly one translation unit, and so it can emit the vtable and RTTI in that
  translation unit only instead of instantiating those in each translation unit
  and given \cc{linkonce_odr} linkage.

  This technique is unfortunate as it relies on detailed knowledge of how common
  toolchains work, and it may also require creating a dummy virtual function.
\end{enumerate}

\section{Open questions}
Does it make sense to declare the RTTI in a separate TU than the vtable?
I think the Itanium ABI requires those to be in the same TU.

\section{Proposed wording}
Will be provided on request.

\section{Acknowledgments}
Thanks to John McCall for the initial idea.

\section{References}
\renewcommand{\section}[2]{}%
\begin{thebibliography}{9}

\bibitem[N4762]{N4762}
  Richard Smith,
  \emph{Working Draft, Standard for Programming Language C++}\newline
  \url{http://www.open-std.org/jtc1/sc22/wg21/docs/papers/2018/n4762.pdf}

\end{thebibliography}

\end{document}
