\documentclass{wg21}

\usepackage{xcolor}
\usepackage{soul}
\usepackage{ulem}
\usepackage{fullpage}
\usepackage{parskip}
\usepackage{csquotes}
\usepackage{listings}
\usepackage{minted}
\usepackage{enumitem}

\lstdefinestyle{base}{
  language=c++,
  breaklines=false,
  basicstyle=\ttfamily\color{black},
  moredelim=**[is][\color{green!50!black}]{@}{@},
  escapeinside={(*@}{@*)}
}

\newcommand{\cc}[1]{\mintinline{c++}{#1}}
\newminted[cpp]{c++}{}


\title{Making \cc{std::vector} constexpr}
\docnumber{P1004R2}
\audience{LWG}
\author{Louis Dionne}{ldionne@apple.com}

\begin{document}
\maketitle

\section{Revision history}
\begin{itemize}
  \item R0 -- Initial draft
  \item R1 -- \begin{itemize}
              \item Per LEWG guidance in RAP, specify that \cc{std::vector}'s
                    iterators are constexpr iterators, as defined in \cite{P0858R0}.
              \item Remove an easter egg from the wording -- I don't mess
                    with LWG.
              \item Other minor fixes from LWG Batavia meeting.
              \end{itemize}
  \item R2 -- \begin{itemize}
              \item Add feature-test macro \cc{__cpp_lib_constexpr_vector}
              \item Add a note to the editor to clarify the intent of the wording
              \item Editorial: fix placement of \cc{[[nodiscard]]} with \cc{constexpr}
              \end{itemize}
\end{itemize}


\section{Abstract}
\cc{std::vector} is not currently \cc{constexpr} friendly. With the loosening
of requirements on \cc{constexpr} in \cite{P0784R1} and related papers, we
can now make \cc{std::vector} \cc{constexpr}, and we should in order to support
the \cc{constexpr} reflection effort (and other evident use cases).


\section{Encountered issues}
We surveyed the implementation of \cc{std::vector} in libc++ and noted the
following issues:
\begin{itemize}
  \item ASAN and debug annotations (like iterator invalidation checks) can't
        work in \cc{constexpr}.
  \item Assertions won't work in \cc{constexpr}.
  \item \cc{pointer_traits<T*>::pointer_to} is used but is not currently
        \cc{constexpr}.
  \item Try-catch blocks are used in some places (e.g. \cc{std::vector::insert}),
        but those can't appear in \cc{constexpr}.
  \item Note that making \cc{std::swap} \cc{constexpr} is not a problem since
        the resolution of \cite{P0859R0}, according to Richard Smith.
\end{itemize}

Assertion and ASAN annotations can be handled by having a mechanism to detect
when a function is evaluated as part of a constant expression, as proposed in
\cite{P0595R0}.

\cc{std::pointer_traits} can be made \cc{constexpr} in the cases we care about;
this is handled by P1006, which should be published in the same mailing as this
paper.

Try-catch blocks could be allowed inside \cc{constexpr}; this is handled by
P1002, which should be published in the same mailing as this paper.


\section{Proposed wording}
This wording is based on the working draft \cite{N4727}. We basically mark
all the member and non-member functions of \cc{std::vector} \cc{constexpr}.

Direction to the editor: please apply \cc{constexpr} to all of \cc{std::vector},
including any additions that might be missing from this paper.

In \textbf{[support.limits.general]}, add the new feature test macro
\cc{__cpp_lib_constexpr_vector} with the corresponding value for header
\cc{<vector>} to Table 36 \textbf{[tab:support.ft]}.

Change in \textbf{[vector.syn] 21.3.6}:
\begin{quote}
\begin{codeblock}
#include <initializer_list>

namespace std {
  // 21.3.11, class template \tcode{vector}
  template<class T, class Allocator = allocator<T>> class vector;

  template<class T, class Allocator>
    @\added{constexpr}@ bool operator==(const vector<T, Allocator>& x, const vector<T, Allocator>& y);
  template<class T, class Allocator>
    @\added{constexpr}@ bool operator!=(const vector<T, Allocator>& x, const vector<T, Allocator>& y);
  template<class T, class Allocator>
    @\added{constexpr}@ bool operator< (const vector<T, Allocator>& x, const vector<T, Allocator>& y);
  template<class T, class Allocator>
    @\added{constexpr}@ bool operator> (const vector<T, Allocator>& x, const vector<T, Allocator>& y);
  template<class T, class Allocator>
    @\added{constexpr}@ bool operator>=(const vector<T, Allocator>& x, const vector<T, Allocator>& y);
  template<class T, class Allocator>
    @\added{constexpr}@ bool operator<=(const vector<T, Allocator>& x, const vector<T, Allocator>& y);

  template<class T, class Allocator>
    @\added{constexpr}@ void swap(vector<T, Allocator>& x, vector<T, Allocator>& y)
      noexcept(noexcept(x.swap(y)));

  [...]
}
\end{codeblock}
\end{quote}

Add after \textbf{[vector.overview] 21.3.11.1/2}:
\begin{quote}
\added{The types \texttt{iterator} and \texttt{const_iterator} meet the
constexpr iterator requirements ([iterator.requirements.general]).}
\end{quote}

Change in \textbf{[vector.overview] 21.3.11.1}:
\begin{quote}
\begin{codeblock}
namespace std {
  template<class T, class Allocator = allocator<T>>
  class vector {
  public:
    // types
    using value_type             = T;
    using allocator_type         = Allocator;
    using pointer                = typename allocator_traits<Allocator>::pointer;
    using const_pointer          = typename allocator_traits<Allocator>::const_pointer;
    using reference              = value_type&;
    using const_reference        = const value_type&;
    using size_type              = @\impdef@; // see 21.2
    using difference_type        = @\impdef@; // see 21.2
    using iterator               = @\impdef@; // see 21.2
    using const_iterator         = @\impdef@; // see 21.2
    using reverse_iterator       = std::reverse_iterator<iterator>;
    using const_reverse_iterator = std::reverse_iterator<const_iterator>;

    // 21.3.11.2, construct/copy/destroy
    @\added{constexpr}@ vector() noexcept(noexcept(Allocator())) : vector(Allocator()) { }
    @\added{constexpr}@ explicit vector(const Allocator&) noexcept;
    @\added{constexpr}@ explicit vector(size_type n, const Allocator& = Allocator());
    @\added{constexpr}@ vector(size_type n, const T& value, const Allocator& = Allocator());
    template<class InputIterator>
      @\added{constexpr}@ vector(InputIterator first, InputIterator last, const Allocator& = Allocator());
    @\added{constexpr}@ vector(const vector& x);
    @\added{constexpr}@ vector(vector&&) noexcept;
    @\added{constexpr}@ vector(const vector&, const Allocator&);
    @\added{constexpr}@ vector(vector&&, const Allocator&);
    @\added{constexpr}@ vector(initializer_list<T>, const Allocator& = Allocator());
    @\added{constexpr}@ ~vector();
    @\added{constexpr}@ vector& operator=(const vector& x);
    @\added{constexpr}@ vector& operator=(vector&& x)
      noexcept(allocator_traits<Allocator>::propagate_on_container_move_assignment::value ||
               allocator_traits<Allocator>::is_always_equal::value);
    @\added{constexpr}@ vector& operator=(initializer_list<T>);
    template<class InputIterator>
      @\added{constexpr}@ void assign(InputIterator first, InputIterator last);
    @\added{constexpr}@ void assign(size_type n, const T& u);
    @\added{constexpr}@ void assign(initializer_list<T>);
    @\added{constexpr}@ allocator_type get_allocator() const noexcept;

    // iterators
    @\added{constexpr}@ iterator               begin() noexcept;
    @\added{constexpr}@ const_iterator         begin() const noexcept;
    @\added{constexpr}@ iterator               end() noexcept;
    @\added{constexpr}@ const_iterator         end() const noexcept;
    @\added{constexpr}@ reverse_iterator       rbegin() noexcept;
    @\added{constexpr}@ const_reverse_iterator rbegin() const noexcept;
    @\added{constexpr}@ reverse_iterator       rend() noexcept;
    @\added{constexpr}@ const_reverse_iterator rend() const noexcept;

    @\added{constexpr}@ const_iterator         cbegin() const noexcept;
    @\added{constexpr}@ const_iterator         cend() const noexcept;
    @\added{constexpr}@ const_reverse_iterator crbegin() const noexcept;
    @\added{constexpr}@ const_reverse_iterator crend() const noexcept;

    // 21.3.11.3, capacity
    [[nodiscard]] @\added{constexpr}@ bool empty() const noexcept;
    @\added{constexpr}@ size_type size() const noexcept;
    @\added{constexpr}@ size_type max_size() const noexcept;
    @\added{constexpr}@ size_type capacity() const noexcept;
    @\added{constexpr}@ void      resize(size_type sz);
    @\added{constexpr}@ void      resize(size_type sz, const T& c);
    @\added{constexpr}@ void      reserve(size_type n);
    @\added{constexpr}@ void      shrink_to_fit();

    // element access
    @\added{constexpr}@ reference       operator[](size_type n);
    @\added{constexpr}@ const_reference operator[](size_type n) const;
    @\added{constexpr}@ const_reference at(size_type n) const;
    @\added{constexpr}@ reference       at(size_type n);
    @\added{constexpr}@ reference       front();
    @\added{constexpr}@ const_reference front() const;
    @\added{constexpr}@ reference       back();
    @\added{constexpr}@ const_reference back() const;

    // 21.3.11.4, data access
    @\added{constexpr}@ T*       data() noexcept;
    @\added{constexpr}@ const T* data() const noexcept;

    // 21.3.11.5, modifiers
    template<class... Args> @\added{constexpr}@ reference emplace_back(Args&&... args);
    @\added{constexpr}@ void push_back(const T& x);
    @\added{constexpr}@ void push_back(T&& x);
    @\added{constexpr}@ void pop_back();

    template<class... Args> @\added{constexpr}@ iterator emplace(const_iterator position, Args&&... args);
    @\added{constexpr}@ iterator insert(const_iterator position, const T& x);
    @\added{constexpr}@ iterator insert(const_iterator position, T&& x);
    @\added{constexpr}@ iterator insert(const_iterator position, size_type n, const T& x);
    template<class InputIterator>
    @\added{constexpr}@ iterator insert(const_iterator position, InputIterator first, InputIterator last);
    @\added{constexpr}@ iterator insert(const_iterator position, initializer_list<T> il);
    @\added{constexpr}@ iterator erase(const_iterator position);
    @\added{constexpr}@ iterator erase(const_iterator first, const_iterator last);
    @\added{constexpr}@ void     swap(vector&)
      noexcept(allocator_traits<Allocator>::propagate_on_container_swap::value ||
               allocator_traits<Allocator>::is_always_equal::value);
    @\added{constexpr}@ void     clear() noexcept;
  };

  template<class InputIterator,
           class Allocator = allocator<@\textit{iter-value-type<InputIterator>}@>>
    vector(InputIterator, InputIterator, Allocator = Allocator())
      -> vector<@\textit{iter-value-type<InputIterator>}@, Allocator>;

  // swap
  template<class T, class Allocator>
    @\added{constexpr}@ void swap(vector<T, Allocator>& x, vector<T, Allocator>& y)
      noexcept(noexcept(x.swap(y)));
}
\end{codeblock}%
\end{quote}

Change in \textbf{[vector.cons] 21.3.11.2}:
\begin{quote}
\begin{itemdecl}
@\added{constexpr}@ explicit vector(const Allocator&);
\end{itemdecl}
[...]
\begin{itemdecl}
@\added{constexpr}@ explicit vector(size_type n, const Allocator& = Allocator());
\end{itemdecl}
[...]
\begin{itemdecl}
@\added{constexpr}@ vector(size_type n, const T& value,
                           const Allocator& = Allocator());
\end{itemdecl}
[...]
\begin{itemdecl}
template<class InputIterator>
  @\added{constexpr}@ vector(InputIterator first, InputIterator last,
                             const Allocator& = Allocator());
\end{itemdecl}
[...]
\end{quote}

Change in \textbf{[vector.capacity] 21.3.11.3}:
\begin{quote}
\begin{itemdecl}
@\added{constexpr}@ size_type capacity() const noexcept;
\end{itemdecl}
[...]
\begin{itemdecl}
@\added{constexpr}@ void reserve(size_type n);
\end{itemdecl}
[...]
\begin{itemdecl}
@\added{constexpr}@ void shrink_to_fit();
\end{itemdecl}
[...]
\begin{itemdecl}
@\added{constexpr}@ void swap(vector& x)
  noexcept(allocator_traits<Allocator>::propagate_on_container_swap::value ||
           allocator_traits<Allocator>::is_always_equal::value);
\end{itemdecl}
[...]
\begin{itemdecl}
@\added{constexpr}@ void resize(size_type sz);
\end{itemdecl}
[...]
\begin{itemdecl}
@\added{constexpr}@ void resize(size_type sz, const T& c);
\end{itemdecl}
[...]
\end{quote}

Change in \textbf{[vector.data] 21.3.11.4}:
\begin{quote}
\begin{itemdecl}
@\added{constexpr}@ T*         data() noexcept;
@\added{constexpr}@ const T*   data() const noexcept;
\end{itemdecl}
\end{quote}

Change in \textbf{[vector.modifiers] 21.3.11.5}:
\begin{quote}
\begin{itemdecl}
@\added{constexpr}@ iterator insert(const_iterator position, const T& x);
@\added{constexpr}@ iterator insert(const_iterator position, T&& x);
@\added{constexpr}@ iterator insert(const_iterator position, size_type n, const T& x);
template<class InputIterator>
  @\added{constexpr}@ iterator insert(const_iterator position, InputIterator first, InputIterator last);
@\added{constexpr}@ iterator insert(const_iterator position, initializer_list<T>);

template<class... Args> @\added{constexpr}@ reference emplace_back(Args&&... args);
template<class... Args> @\added{constexpr}@ iterator emplace(const_iterator position, Args&&... args);
@\added{constexpr}@ void push_back(const T& x);
@\added{constexpr}@ void push_back(T&& x);
\end{itemdecl}
[...]
\begin{itemdecl}
@\added{constexpr}@ iterator erase(const_iterator position);
@\added{constexpr}@ iterator erase(const_iterator first, const_iterator last);
@\added{constexpr}@ void pop_back();
\end{itemdecl}
\end{quote}

Change in \textbf{[vector.special] 21.3.11.6}:
\begin{quote}
\begin{itemdecl}
template<class T, class Allocator>
  @\added{constexpr}@ void swap(vector<T, Allocator>& x, vector<T, Allocator>& y)
    noexcept(noexcept(x.swap(y)));
\end{itemdecl}
\end{quote}

Change in \textbf{[vector.bool] 21.3.12/1}:
\begin{quote}
To optimize space allocation, a specialization of vector for \tcode{bool}
elements is provided:

\begin{codeblock}
namespace std {
  template<class Allocator>
  class vector<bool, Allocator> {
  public:
    // types
    using value_type             = bool;
    using allocator_type         = Allocator;
    using pointer                = @\impdef@;
    using const_pointer          = @\impdef@;
    using const_reference        = bool;
    using size_type              = @\impdef@; // see 21.2
    using difference_type        = @\impdef@; // see 21.2
    using iterator               = @\impdef@; // see 21.2
    using const_iterator         = @\impdef@; // see 21.2
    using reverse_iterator       = std::reverse_iterator<iterator>;
    using const_reverse_iterator = std::reverse_iterator<const_iterator>;

    // bit reference
    class reference {
      friend class vector;
      @\added{constexpr}@ reference() noexcept;
    public:
      @\added{constexpr}@ ~reference();
      @\added{constexpr}@ reference(const reference&) = default;
      @\added{constexpr}@ operator bool() const noexcept;
      @\added{constexpr}@ reference& operator=(const bool x) noexcept;
      @\added{constexpr}@ reference& operator=(const reference& x) noexcept;
      @\added{constexpr}@ void flip() noexcept;     // flips the bit
    };

    // construct/copy/destroy
    @\added{constexpr}@ vector() : vector(Allocator()) { }
    @\added{constexpr}@ explicit vector(const Allocator&);
    @\added{constexpr}@ explicit vector(size_type n, const Allocator& = Allocator());
    @\added{constexpr}@ vector(size_type n, const bool& value, const Allocator& = Allocator());
    template<class InputIterator>
      @\added{constexpr}@ vector(InputIterator first, InputIterator last, const Allocator& = Allocator());
    @\added{constexpr}@ vector(const vector& x);
    @\added{constexpr}@ vector(vector&& x);
    @\added{constexpr}@ vector(const vector&, const Allocator&);
    @\added{constexpr}@ vector(vector&&, const Allocator&);
    @\added{constexpr}@ vector(initializer_list<bool>, const Allocator& = Allocator()));
    @\added{constexpr}@ ~vector();
    @\added{constexpr}@ vector& operator=(const vector& x);
    @\added{constexpr}@ vector& operator=(vector&& x);
    @\added{constexpr}@ vector& operator=(initializer_list<bool>);
    template<class InputIterator>
      @\added{constexpr}@ void assign(InputIterator first, InputIterator last);
    @\added{constexpr}@ void assign(size_type n, const bool& t);
    @\added{constexpr}@ void assign(initializer_list<bool>);
    @\added{constexpr}@ allocator_type get_allocator() const noexcept;

    // iterators
    @\added{constexpr}@ iterator               begin() noexcept;
    @\added{constexpr}@ const_iterator         begin() const noexcept;
    @\added{constexpr}@ iterator               end() noexcept;
    @\added{constexpr}@ const_iterator         end() const noexcept;
    @\added{constexpr}@ reverse_iterator       rbegin() noexcept;
    @\added{constexpr}@ const_reverse_iterator rbegin() const noexcept;
    @\added{constexpr}@ reverse_iterator       rend() noexcept;
    @\added{constexpr}@ const_reverse_iterator rend() const noexcept;

    @\added{constexpr}@ const_iterator         cbegin() const noexcept;
    @\added{constexpr}@ const_iterator         cend() const noexcept;
    @\added{constexpr}@ const_reverse_iterator crbegin() const noexcept;
    @\added{constexpr}@ const_reverse_iterator crend() const noexcept;

    // capacity
    [[nodiscard]] @\added{constexpr}@ bool empty() const noexcept;
    @\added{constexpr}@ size_type size() const noexcept;
    @\added{constexpr}@ size_type max_size() const noexcept;
    @\added{constexpr}@ size_type capacity() const noexcept;
    @\added{constexpr}@ void      resize(size_type sz, bool c = false);
    @\added{constexpr}@ void      reserve(size_type n);
    @\added{constexpr}@ void      shrink_to_fit();

    // element access
    @\added{constexpr}@ reference       operator[](size_type n);
    @\added{constexpr}@ const_reference operator[](size_type n) const;
    @\added{constexpr}@ const_reference at(size_type n) const;
    @\added{constexpr}@ reference       at(size_type n);
    @\added{constexpr}@ reference       front();
    @\added{constexpr}@ const_reference front() const;
    @\added{constexpr}@ reference       back();
    @\added{constexpr}@ const_reference back() const;

    // modifiers
    template<class... Args> @\added{constexpr}@ reference emplace_back(Args&&... args);
    @\added{constexpr}@ void push_back(const bool& x);
    @\added{constexpr}@ void pop_back();
    template<class... Args> @\added{constexpr}@ iterator emplace(const_iterator position, Args&&... args);
    @\added{constexpr}@ iterator insert(const_iterator position, const bool& x);
    @\added{constexpr}@ iterator insert(const_iterator position, size_type n, const bool& x);
    template<class InputIterator>
      @\added{constexpr}@ iterator insert(const_iterator position, InputIterator first, InputIterator last);
    @\added{constexpr}@ iterator insert(const_iterator position, initializer_list<bool> il);

    @\added{constexpr}@ iterator erase(const_iterator position);
    @\added{constexpr}@ iterator erase(const_iterator first, const_iterator last);
    @\added{constexpr}@ void swap(vector&);
    @\added{constexpr}@ static void swap(reference x, reference y) noexcept;
    @\added{constexpr}@ void flip() noexcept;       // flips all bits
    @\added{constexpr}@ void clear() noexcept;
  };
}
\end{codeblock}%
\end{quote}

Change in \textbf{[vector.bool] 21.3.12/4}:
\begin{quote}
\begin{itemdecl}
@\added{constexpr}@ void flip() noexcept;
\end{itemdecl}
\end{quote}

Change in \textbf{[vector.bool] 21.3.12/5}:
\begin{quote}
\begin{itemdecl}
@\added{constexpr}@ static void swap(reference x, reference y) noexcept;
\end{itemdecl}
\end{quote}


\section{References}
\renewcommand{\section}[2]{}%
\begin{thebibliography}{9}

\bibitem[N4727]{N4727}
  Richard Smith,
  \emph{Working Draft, Standard for Programming Language C++}\newline
  \url{http://www.open-std.org/jtc1/sc22/wg21/docs/papers/2018/n4727.pdf}

\bibitem[P0784R1]{P0784R1}
  Multiple authors,
  \emph{Standard containers and constexpr}\newline
  \url{http://www.open-std.org/jtc1/sc22/wg21/docs/papers/2018/p0784r1.html}

\bibitem[P0859R0]{P0859R0}
  Richard Smith,
  \emph{Core Issue 1581: When are constexpr member functions defined?}\newline
  \url{http://www.open-std.org/jtc1/sc22/wg21/docs/papers/2017/p0859r0.html}

\bibitem[P0595R0]{P0595R0}
  David Vandevoorde,
  \emph{The constexpr Operator}\newline
  \url{http://www.open-std.org/jtc1/sc22/wg21/docs/papers/2017/p0595r0.html}

\bibitem[P0858R0]{P0858R0}
  Antony Polukhin,
  \emph{Constexpr iterator requirements}\newline
  \url{http://www.open-std.org/jtc1/sc22/wg21/docs/papers/2018/p0858r0.html}

\end{thebibliography}

\end{document}
