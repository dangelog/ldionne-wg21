\documentclass[11pt]{article}

\usepackage{xcolor}
\usepackage{soul}
\usepackage{ulem}
\usepackage{fullpage}
\usepackage[colorlinks, allcolors=blue]{hyperref}
\usepackage{parskip}
\usepackage{csquotes}
\usepackage{listings}
\usepackage{minted}

\lstdefinestyle{base}{
  language=c++,
  breaklines=false,
  basicstyle=\ttfamily\color{black},
  moredelim=**[is][\color{green!50!black}]{@}{@},
  escapeinside={(*@}{@*)}
}

\newcommand{\cc}[1]{\mintinline{c++}{#1}}
\newminted[cpp]{c++}{}
\newcommand{\deleted}[1]{\st{#1}}
\newcommand{\added}[1]{\sethlcolor{green}{\hl{#1}}}

\date{}
\title{Lambdas in unevaluated contexts}
\author{}


\begin{document}

\maketitle\vspace{-2cm}

\begin{flushright}
  \begin{tabular}{ll}
  Document \#:&D0315R1\\
  Date:       &\date{2016-08-01}\\
  Project:    &Programming Language C++\\
              &Evolution Group\\
  Reply-to:   &\author{Louis Dionne} \textless\href{mailto:ldionne.2@gmail.com}{ldionne.2@gmail.com}\textgreater
  \end{tabular}
\end{flushright}

\section{Revision history}
\begin{itemize}
  \item R0 -- Initial draft
  \item R1 -- Changed the wording to work around the resolution of \cite{DR1607},
              which conflicted with the initial wording. Also address the
              potential additional concerns raised by this wording change.
\end{itemize}


\section{Introduction}
Lambdas are a very powerful language feature, especially when it comes to using
higher-order algorithms with custom predicates or expressing small, disposable
pieces of code. Yet, they suffer from one important limitation which cripples
their usefulness for creative use cases; they can't appear in unevaluated
contexts. This restriction was originally designed to prevent lambdas from
appearing in signatures, which would have opened a can of worm for mangling
because lambdas are required to have unique types. However, the restriction is
much stronger than it needs to be, and it is indeed possible to achieve the
same effect without it, as evidenced by this paper.


\section{Motivation}
The original use case that motivated this article is related to making algorithms
on heterogeneous containers more useful. For a bit of background, it is possible
to implement \cc{std}-like algorithms that operate on \cc{std::tuple}s instead of
usual, runtime sequences. For example, it is possible to write an algorithm akin
to \cc{std::sort}, but which works on a \cc{std::tuple} instead of a runtime
sequence:

\begin{cpp}
// Returns a new tuple whose elements are sorted according to the given
// binary predicate, which must return a boolean `std::integral_constant`.
template <typename ...T, typename Predicate>
auto sort(std::tuple<T...> const& tuple, Predicate const& pred);
\end{cpp}

The algorithm can then be used as follows:

\begin{cpp}
auto tuple = std::make_tuple(std::array<int, 5>{}, 1, '2', 3.3);
auto sorted = sort(tuple, [](auto const& a, auto const& b) {
  return std::integral_constant<bool, sizeof(a) < sizeof(b)>{};
});
// sorted is now a std::tuple<char, int, double, std::array<int, 5>>
\end{cpp}

While this is a simplified example, it is also possible to define other algorithms
like \cc{for_each}, \cc{transform}, \cc{accumulate}, \cc{find_if} and many more. This
is exploited extensively in the \cite{Boost.Hana} library, which provides high-level
algorithms and data structures to make metaprogramming more structured.

Where the current proposal meets with the above use case is when one needs the type
resulting from an algorithm exposed above. For example, to get the type of the above
tuple without actually creating the tuple, one would like to simply write

\begin{cpp}
using sorted = decltype(sort(tuple, [](auto const& a, auto const& b) {
  return std::integral_constant<bool, sizeof(a) < sizeof(b)>{};
}));
\end{cpp}

Unfortunately, with the current restriction on lambdas, this is impossible.
Instead, one must create a variable holding the lambda, and then pass this
variable to the algorithm:

\begin{cpp}
auto predicate = [](auto const& a, auto const& b) {
  return std::integral_constant<bool, sizeof(a) < sizeof(b)>{};
};
using sorted = decltype(sort(tuple, predicate));
\end{cpp}

Unfortunately, this is both clumsy and not always possible since some contexts
do not allow defining local variables (for example inside a class declaration).
Hence, the restriction severly reduces the usefulness of lambdas in these algorithms.
Also note that the issue presented above does not only arise in the context of
manipulating heterogeneous containers. Indeed, one could just as well try to
write the following, only to be puzzled by a compiler error:

\begin{cpp}
std::vector<int> v{1,2,3,4};
using Iterator = decltype(std::find_if(begin(v), end(v), [](int i) {
  return i % 2 == 0;
}));
\end{cpp}

While this is a valid use case, it is expected that using \cc{decltype} on
such a complex expression is less frequent outside the realm of heterogeneous
computations.

Finally, another motivation for this paper is that the restriction is much stronger
than it needs to be, and it prevents lambdas from being used in creative ways,
some of which are certainly unknown to the author of this paper.


\section{Proposed Wording}
The wording changes introduced in this version of the paper are much greater in
scope than that of the original paper. The reason for this is that we remove
many restrictions on lamdba expressions, yet we still want to keep closure types
out of the signature of external functions (which would be a nightmare for
implementations). Thus, much of the wording presented below clarifies the
standard to handle possible concerns with closure types sneaking into signatures.
The wording is based on the working paper \cite{N4606}:

\begin{enumerate}
  \item First, we remove both the restriction that lambda expressions may not
  appear in unevaluated contexts and the restrictions added by the resolution
  of \cite{DR1607}, which would otherwise prevent our use cases.
  In \textbf{[expr.prim.lambda] 5.1.2/2} modify the paragraph as follows:
  \begin{quote}
    The evaluation of a lambda-expression results in a prvalue temporary
    (12.2). This temporary is called the closure object. \deleted{A
    lambda-expression shall not appear in an unevaluated operand (Clause 5),
    in a template-argument, in an alias-declaration, in a typedef declaration,
    or in the declaration of a function or function template outside its function
    body and default arguments. [ Note: The intention is to prevent lambdas from
    appearing in a signature. -- end note ]} [ Note: A closure object behaves like
    a function object (20.9). -- end note ]
  \end{quote}


  \item With the removal of the above restrictions, a concern is that lambda-
  expressions might then be able to appear in the signature of functions with
  external linkage, which would require implementations to generate a name for
  the associated closure type. However, since we wouldn't be able to attach to
  another ABI entity in some cases (such as lambda-expressions appearing at
  global scope), that would mean coming up with a mangling scheme that identifies
  the closure type from nothing but its own form. This, in turn, would require
  encoding its complete definition, which is burdensome for implementations
  and motivated the original restrictions on lambda-expressions.

  Fortunately, this specific problem can't arise in the case of non-template
  functions, even with the above removal of constraints on lambda-expressions.
  Indeed, according to \textbf{[basic.link] 3.5/8}, closure types have no linkage,
  and therefore they cannot appear in the signature of a function with external
  linkage (a function is a compound type):
  \begin{quote}
    \textit{[...]} A type is said to have linkage if and only if:\\
    \textit{[...]}\\
    - it is a compound type (3.9.2) other than a class or enumeration,
      compounded exclusively from types that have linkage; or \\
    \textit{[...]}
  \end{quote}

  However, to make it clear that closure types are never given a name for
  linkage purposes, we propose modifying \textbf{[decl.typedef] 7.1.3/9} as
  follows:
  \begin{quote}
    If the typedef declaration defines an unnamed class (or enum), the first
    typedef-name declared by the declaration to be that class type (or enum type)
    is used to denote the class type (or enum type) for linkage purposes only (3.5).
    \added{However, a closure type is never given a name for linkage purposes.}
    [ \textit{Example:}
\begin{lstlisting}[style=base]
typedef struct { } *ps, S; // S is the class name for linkage purposes
(*@\added{typedef decltype([]\{\}) C; // the closure type has no name for linkage purposes}@*)
\end{lstlisting}
    \textit{-- end example} ]
  \end{quote}


  \item Another similar problem is that of lambda-expressions appearing in the
  signature of function templates, not by themselves but indirectly by being
  part of an expression which references a template parameter. Indeed, per
  \textbf{[temp.over.link] 14.5.6.1/4}:
  \begin{quote}
    When an expression that references a template parameter is used in the
    function parameter list or the return type in the declaration of a function
    template, the expression that references the template parameter is part of
    the signature of the function template.
  \end{quote}

  Thus, a function template declaration such as the following will require the
  implementation to make the lambda-expression part of the signature, which is
  specifically what we would like to avoid:
\begin{cpp}
template <unsigned N>
void f(const char (*s)[([]() { return N; })()]) { }
\end{cpp}

  To make sure this does not happen, we propose amending \textbf{[temp.over.link]
  14.5.6.1/5} as follows:
  \begin{quote}
    Two expressions involving template parameters are considered equivalent
    if two function definitions containing the expressions would satisfy the
    one-definition rule (3.2), except that the tokens used to name the template
    parameters may differ as long as a token used to name a template parameter
    in one expression is replaced by another token that names the same template
    parameter in the other expression. \added{Two lambda-expressions appearing
    in full expressions involving template parameters are never considered
    equivalent. [ \textit{Note:} The intent is to avoid lambda-expressions
    appearing in the signature of a function template with external linkage.
    \textit{-- end note} ]}
  \end{quote}

  Furthermore, if more clarity is desired, the following example could be added
  after \textbf{[temp.over.link] 14.5.6.1/5}:
\begin{cpp}
// The program is ill-formed, because the two lambda-expressions are
// functionally equivalent but not equivalent.
// a.cc
template <unsigned N>
void foo(const char (*s)[([]() { return N; })()]);

// b.cc
template <unsigned N>
void foo(const char (*s)[([]() { return N; })()]);
\end{cpp}


  \item Another possible concern is the appearance of lambda-expressions in
  contexts that are constrained by the ODR. For example:

\begin{cpp}
// a.h:
template <typename T>
int counter() {
  static int cnt = 0;
  return cnt++;
}

inline int f() {
  return counter<decltype([] {})>();
}

// t0.cc:
#include "a.h"
int foo() { return f(); }

// t1.cc:
#include "a.h"
int bar() { return f(); }
\end{cpp}

  Given such code, a question might be whether \cc foo and \cc bar modify
  the same \cc cnt variable, since \cc f is defined in a header and it calls
  \cc counter with a closure type that is supposed to be unique. However,
  since \cc f is \cc inline, the resulting program is as-if there was a
  single definition of it, and so both functions end up modifying the same
  \cc cnt variable. This turns out not to be a problem for implementations,
  because they must already handle such cases where there is an ODR context
  to attach the closure type to. Thus, no wording change is required.


  \item A concern with allowing lambda-expressions in \textit{declarations} is
  that of dealing with entities that can be redeclared. However, we feel like
  no \textit{normative} change to the wording is required to address that concern.
  Indeed, it is already the case that no two lambda-expressions share the same
  closure type within a single translation unit. By further clarifying that the
  lambda-expressions in alias template specializations are unique to each
  specialization even if non-dependent, we conclude the following:

\begin{cpp}
static decltype([] { }) f();
static decltype([] { }) f(); // invalid; return type mismatch

static void g(decltype([] { }) *) { }
static void g(decltype([] { }) *) { }
g(nullptr); // ambiguous

using A = decltype([] { });
static void h(A *);
static void h(A *) { }
h(nullptr); // okay

template <typename T>
using B = decltype([] { });
static void i(B<char16_t> *) { }
static void i(B<char32_t> *) { }
i(nullptr); // ambiguous
\end{cpp}

  To make the above interpretation of the standard more obvious, we propose
  adding the following paragraph at the end of \textbf{[temp.alias] 14.5.7}:
  \begin{quote}
    The type of a lambda expression appearing in an alias template declaration
    is unique across instantiations of that alias template, even when the
    lambda expression is not dependent.
    [ \textit{Example:}
\begin{cpp}
template <typename T>
using A = decltype([] {});
// A<int> and A<char> refer to different closure types
\end{cpp}
    \textit{-- end example} ]
  \end{quote}


  \item A concern with allowing lambda-expressions \textit{outside the body of
  the declaration of function templates} is the need to evaluate the validity
  of potentially complex expressions as part of template argument deduction.
  Indeed, without clarifying the wording, it is unclear whether implementations
  would be expected to support \textit{SFINAE} based on the validity of
  the body of a lambda-expression found in the declaration of a function
  template. Since this could be unwieldy for implementations, we choose not
  to require this in the current paper. Thus, if a lambda-expression appears
  inside the declaration of a function template and the body of that lambda
  is ill-formed, then the program is ill-formed. To reflect this, we add the
  following note after \textbf{[temp.deduct] 14.8.2/8} (note that the term
  \textit{immediate context} is not defined formally in the standard, which
  is the subject of \cite{CWG1844}):

  \begin{quote}
    [ \textit{Note:} The body of a lambda expression appearing in a function
    type or a template parameter is not considered part of the immediate
    context for the purposes of template argument deduction. The intent is
    to avoid requiring implementations to deal with substitution failure
    involving arbitrary statements. [ \textit{Example:}
\begin{cpp}
template <typename T>
auto f(T) -> decltype([]() { T::invalid; } ());
void f(...);
f(0); // invalid expression not part of the immediate context, hard error

template <typename T, std::size_t = sizeof([]() { T::invalid; })>
void g(T);
void g(...);
g(0); // invalid expression not part of the immediate context, hard error

template <typename T>
auto h(T) -> decltype([]() -> std::void_t<typename T::invalid> { });
void h(...);
h(0); // deduction fails on #1, calls #2

template <typename T>
auto i(T) -> decltype([](){ } (t));
void i(...);
i(0); // deduction fails on #1, calls #2
\end{cpp}
    \textit{-- end example} ]
    \textit{-- end note} ]
  \end{quote}

\end{enumerate}


\section{Implementation Experience}
This proposal was implemented naively in Clang. The required change is commenting
a single line which creates a diagnostic if a lambda-expression is found inside an
unevaluated context.


\section{Acknowledgements}
Roland Bock and Matt Calabrese for discussing use cases for lambdas in unevaluated
contexts on the \textit{std-proposal} and \textit{Boost.Devel} mailing lists.
Richard Smith for letting me know that we could do without the restriction.
David Vandervoorde and Hubert Tong for providing extensive guidance in the
second iteration of the wording.


\section{References}
\renewcommand{\section}[2]{}%
\begin{thebibliography}{9}

  \bibitem[Boost.Hana]{Boost.Hana}
    Louis Dionne,
    \emph{Boost.Hana, A modern metaprogramming library}\newline
    \url{https://github.com/boostorg/hana}

  \bibitem[N4606]{N4606}
    Richard Smith,
    \emph{Working Draft, Standard for Programming Language C++}\newline
    \url{https://github.com/cplusplus/draft/blob/master/papers/n4606.pdf}

  \bibitem[DR1607]{DR1607}
    Daniel Kr{\"u}gler,
    \emph{Lambdas in template parameters}\newline
    \url{http://www.open-std.org/jtc1/sc22/wg21/docs/cwg_defects.html#1607}

  \bibitem[CWG1844]{CWG1844}
    Richard Smith,
    \emph{Defining “immediate context”}\newline
    \url{http://www.open-std.org/jtc1/sc22/wg21/docs/cwg_active.html#1844}

\end{thebibliography}

\end{document}
