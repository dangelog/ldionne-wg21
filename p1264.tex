\documentclass{wg21}

\usepackage[T1]{fontenc}
\usepackage[utf8]{inputenc}

\usepackage{csquotes}
\usepackage{enumitem}
\usepackage{fullpage}
\usepackage{hyperref}
\usepackage{listings}
\usepackage{minted}
\usepackage{parskip}
\usepackage{soul}
\usepackage{ulem}
\usepackage{xcolor}

\lstdefinestyle{base}{
  language=c++,
  breaklines=false,
  basicstyle=\ttfamily\color{black},
  moredelim=**[is][\color{green!50!black}]{@}{@},
  escapeinside={(*@}{@*)}
}

\newcommand{\cc}[1]{\mintinline{c++}{#1}}
\newminted[cpp]{c++}{}


\title{Revising the wording of stream input operations}
\docnumber{D1264R1}
\audience{LWG}
\author{Louis Dionne}{ldionne@apple.com}

\begin{document}
\maketitle

\section{Revision history}
\begin{itemize}
  \item R0 -- Initial draft
  \item R1 -- \begin{itemize}
              \item Apply LWG small-group changes.
              \item Rebase on top of N4778.
              \end{itemize}
\end{itemize}

\section{Abstract}
The wording in \textbf{[istream]}, \textbf{[istream.formatted]} and
\textbf{[istream.unformatted]} is very difficult to follow when it comes
to exceptions. Some requirements are specified more than once in different
locations, which makes it ambiguous how requirements should interact with
each other.

This is problematic because implementations currently differ significantly
on their handling of error flags and exceptions. For example,
\href{https://bugs.llvm.org/show_bug.cgi?id=21586}{this} libc++ bug
report claims that libc++'s \break \cc{operator>>(istream&, std::string&)} is
not throwing exception when \cc{failbit} is set and \cc{failbit} exceptions
are enabled. GCC and MSVC both behave as expected there. Unfortunately, the
Standard seems to give reason to libc++, despite the behavior not making sense.

\cite{LWG2349} tries to solve this issue by applying a small patch to the current
wording, but I think not all issues are solved this way. This wording-only
proposal instead tries to overhaul the current wording to make it clearer,
without changing the intended behavior.


\section{Proposed wording}
This wording is based on the working draft \cite{N4778}.

\subsection{Remove common wording}
First, we clean up confusing wording that overlaps with wording in the formatted
and unformatted input operations:

Remove \textbf{[istream] 27.7.4.1/3}
\begin{quote}
\removed{If \texttt{rdbuf()->sbumpc()} or \texttt{rdbuf()->sgetc()} returns \texttt{traits::eof()},
then the input function, except as explicitly noted otherwise, completes its
actions and does \texttt{setstate(eofbit)}, which may throw \texttt{ios_base::failure}
(27.5.5.4), before returning.}
\end{quote}

Remove \textbf{[istream] 27.7.4.1/4}:
\begin{quote}
\removed{If one of these called functions throws an exception, then unless explicitly noted
otherwise, the input function sets \texttt{badbit} in error state. If \texttt{badbit}
is on in \texttt{exceptions()}, the input function rethrows the exception without
completing its actions, otherwise it does not throw anything and proceeds as if
the called function had returned a failure indication.}
\end{quote}


\subsection{Revise wording for formatted input operations}
We precise the execution of formatted input operations by introducing the
notion of a \textit{local error state}. In \textbf{[istream.formatted.reqmts] 27.7.4.2.1}:
\begin{quote}
Each formatted input function begins execution by constructing \added{an object
of type \texttt{ios_base::iostate}, called the \textit{local error state}, and
initializing it to \texttt{ios_base::goodbit}. It then creates} an object of
class \texttt{sentry} with the \texttt{noskipws} (second) argument \texttt{false}.
If the sentry object returns \texttt{true}, when converted to a value of type
\texttt{bool}, the function endeavors to obtain the requested input.
\added{Otherwise, if the sentry constructor exits by throwing an exception or
if the sentry object produces \texttt{false} when converted to a value of type
\texttt{bool}, the function returns without attempting to obtain any input.
If an extraction function \href{http://eel.is/c++draft/istream\#2}{(27.7.4.1/2)}
returns \texttt{traits::eof()}, then \texttt{ios_base::eofbit} is turned on in
the \textit{local error state} and the input function stops trying to obtain the requested
input.} If an exception is thrown during input then \texttt{ios::badbit} is
turned on in \added{the \textit{local error state}, \texttt{*this}'s error state is set
to the \textit{local error state}, and the exception is rethrown if
\texttt{(exceptions() \& badbit) != 0}}. \removed{If \texttt{(exceptions()\&badbit) != 0}
then the exception is rethrown.} \added{After extraction is done, the input
function calls \texttt{setstate}, which sets \texttt{*this}'s error state to
the \textit{local error state}, and may throw an exception.} In any case, the
formatted input function destroys the sentry object. If no exception has been
thrown, it returns \texttt{*this}.
\end{quote}

Then, we adjust the description of formatted input operations to take advantage
of the \textit{local error state} introduced above. In \textbf{[istream.formatted.arithmetic]
27.7.4.2.2/1}:
\begin{quote}
\begin{codeblock}
operator>>(unsigned short& val);
operator>>(unsigned int& val);
operator>>(long& val);
operator>>(unsigned long& val);
operator>>(long long& val);
operator>>(unsigned long long& val);
operator>>(float& val);
operator>>(double& val);
operator>>(long double& val);
operator>>(bool& val);
operator>>(void*& val);
\end{codeblock}
As in the case of the inserters, these extractors depend on the locale's
\texttt{num_get<>} (26.4.2.1) object to perform parsing the input stream data.
These extractors behave as formatted input functions (as described in 27.7.4.2.1).
After a sentry object is constructed, the conversion occurs as if performed by
the following code fragment\added{, where \texttt{state} represents the input
function's \textit{local error state}}:

\begin{codeblock}
  using numget = num_get<charT, istreambuf_iterator<charT, traits>>;
  @\removed{iostate err = iostate::goodbit;}@
  use_facet<numget>(loc).get(*this, 0, *this, @\changed{err}{state}@, val);
  @\removed{setstate(err);}@
\end{codeblock}
\end{quote}

In \textbf{[istream.formatted.arithmetic] 27.7.4.2.2/2}:
\begin{quote}
\begin{codeblock}
operator>>(short& val);
\end{codeblock}
The conversion occurs as if performed by the following code fragment (using the
same notation as for the preceding code fragment):
\begin{codeblock}
  using numget = num_get<charT, istreambuf_iterator<charT, traits>>;
  @\removed{iostate err = ios_base::goodbit;}@
  long lval;
  use_facet<numget>(loc).get(*this, 0, *this, @\changed{err}{state}@, lval);
  if (lval < numeric_limits<short>::min()) {
    @\changed{err}{state}@ |= ios_base::failbit;
    val = numeric_limits<short>::min();
  } else if (numeric_limits<short>::max() < lval) {
    @\changed{err}{state}@ |= ios_base::failbit;
    val = numeric_limits<short>::max();
  }  else
    val = static_cast<short>(lval);
  @\added{\}}@
  @\removed{setstate(err);}@
\end{codeblock}
\end{quote}

In \textbf{[istream.formatted.arithmetic] 27.7.4.2.2/3}:
\begin{quote}
\begin{codeblock}
operator>>(int& val);
\end{codeblock}
The conversion occurs as if performed by the following code fragment (using the
same notation as for the preceding code fragment):
\begin{codeblock}
  using numget = num_get<charT, istreambuf_iterator<charT, traits>>;
  @\removed{iostate err = ios_base::goodbit;}@
  long lval;
  use_facet<numget>(loc).get(*this, 0, *this, @\changed{err}{state}@, lval);
  if (lval < numeric_limits<int>::min()) {
    @\changed{err}{state}@ |= ios_base::failbit;
    val = numeric_limits<int>::min();
  } else if (numeric_limits<int>::max() < lval) {
    @\changed{err}{state}@ |= ios_base::failbit;
    val = numeric_limits<int>::max();
  }  else
    val = static_cast<int>(lval);
  @\added{\}}@
  @\removed{setstate(err);}@
\end{codeblock}
\end{quote}

In \textbf{[istream.extractors] 27.7.4.2.3/10}:
\begin{quote}
\begin{codeblock}
template<class charT, class traits>
  basic_istream<charT, traits>& operator>>(basic_istream<charT, traits>& in, charT* s);
template<class traits>
  basic_istream<char, traits>& operator>>(basic_istream<char, traits>& in, unsigned char* s);
template<class traits>
  basic_istream<char, traits>& operator>>(basic_istream<char, traits>& in, signed char* s);
\end{codeblock}
[...]
If the function extracted no characters, \changed{it calls
\texttt{setstate(failbit)}, which may throw \texttt{ios_base::failure} (27.5.5.4)}
{\texttt{ios_base::failbit} is turned on in the input function's \textit{local error state}
before \texttt{setstate} is called}.
\end{quote}

In \textbf{[istream.extractors] 27.7.4.2.3/12}:
\begin{quote}
\begin{codeblock}
template<class charT, class traits>
  basic_istream<charT, traits>& operator>>(basic_istream<charT, traits>& in, charT& c);
template<class traits>
  basic_istream<char, traits>& operator>>(basic_istream<char, traits>& in, unsigned char& c);
template<class traits>
  basic_istream<char, traits>& operator>>(basic_istream<char, traits>& in, signed char& c);
\end{codeblock}
\textit{Effects:} Behaves like a formatted input member (as described in 27.7.4.2.1)
of \texttt{in}. \changed{After a sentry object is constructed a}{A} character
is extracted from \texttt{in}, if one is available, and stored in \texttt{c}.
Otherwise, \changed{the function calls in.setstate(failbit)}{\texttt{ios_base::failbit}
is turned on in the input function's \textit{local error state} before \texttt{setstate} is called}.
\end{quote}

In \textbf{[string.io] 20.3.3.9/1}:
\begin{quote}
\begin{codeblock}
template<class charT, class traits, class Allocator>
  basic_istream<charT, traits>&
    operator>>(basic_istream<charT, traits>& is, basic_string<charT, traits, Allocator>& str);
\end{codeblock}
\textit{Effects:} Behaves as a formatted input function (27.7.4.2.1). After
constructing a sentry object, if the sentry converts to \texttt{true}, calls
\texttt{str.erase()} and then extracts characters from \texttt{is} and appends
them to \texttt{str} as if by calling \texttt{str.append(1, c)}. If \texttt{is.width()}
is greater than zero, the maximum number \texttt{n} of characters appended is
\texttt{is.width()}; otherwise \texttt{n} is \texttt{str.max_size()}.
Characters are extracted and appended until any of the following occurs:
\begin{itemize}
  \item[--] \texttt{n} characters are stored;
  \item[--] end-of-file occurs on the input sequence;
  \item[--] \texttt{isspace(c, is.getloc())} is \texttt{true} for the next
            available input character \texttt{c}.
\end{itemize}
After the last character (if any) is extracted, \texttt{is.width(0)} is called
and the sentry object is destroyed. If the function extracts no characters,
\changed{it calls \texttt{is.setstate(ios::failbit)}, which may throw
\texttt{ios_base::failure} (27.5.5.4)}{\texttt{ios_base::failbit} is turned on
in the input function's \textit{local error state} before \texttt{setstate} is called}.
\end{quote}

In \textbf{[bitset.operators] 19.9.4/4}:
\begin{quote}
\begin{codeblock}
template<class charT, class traits, size_t N>
  basic_istream<charT, traits>&
    operator>>(basic_istream<charT, traits>& is, bitset<N>& x);
\end{codeblock}
A formatted input function (27.7.4.2).

\textit{Effects:} Extracts up to \texttt{N} characters from \texttt{is}. Stores
these characters in a temporary object \texttt{str} of type \texttt{basic_string<charT, traits>},
then evaluates the expression \texttt{x = bitset<N>(str)}. Characters are
extracted and stored until any of the following occurs:
\begin{itemize}
  \item[--] \texttt{N} characters have been extracted and stored;
  \item[--] end-of-file occurs on the input sequence;
  \item[--] the next input character is neither \texttt{is.widen(’0’)} nor
            \texttt{is.widen(’1’)} (in which case the input character is not extracted).
\end{itemize}
If no characters are stored in \texttt{str}, \changed{calls \texttt{is.setstate(ios_base::failbit)}
(which may throw \texttt{ios_base::failure} (27.5.5.4))}{\texttt{ios_base::failbit}
is turned on in the input function's \textit{local error state} before \texttt{setstate} is called}.
\end{quote}


\subsection{Revise wording for unformatted input operations}
In \textbf{[istream.unformatted] 27.7.4.3}:
\begin{quote}
Each unformatted input function begins execution by constructing \added{an
object of type \texttt{ios_base::iostate}, called the \textit{local error state}, and
initializing it to \texttt{ios_base::goodbit}. It then creates} an object of
class \texttt{sentry} with the default argument \texttt{noskipws} (second)
argument \texttt{true}. If the sentry object returns \texttt{true}, when
converted to a value of type \texttt{bool}, the function endeavors to obtain
the requested input. Otherwise, if the sentry constructor exits by throwing an
exception or if the sentry object \changed{returns}{produces} \texttt{false}\removed{,} when converted to a
value of type \texttt{bool}, the function returns without attempting to obtain
any input. In either case the number of extracted characters is set to \texttt{0};
unformatted input functions taking a character array of nonzero size as an
argument shall also store a null character (using \texttt{charT()}) in the
first location of the array. \added{If an extraction function
\href{http://eel.is/c++draft/istream\#2}{(27.7.4.1/2)} returns \texttt{traits::eof()},
then \texttt{ios_base::eofbit} is turned on in the \textit{local error state} and the
input function stops trying to obtain the requested input.} If an exception is
thrown during input then \texttt{ios::badbit} is turned on in \changed{
\texttt{*this}'s error state}{the \textit{local error state}, \texttt{*this}'s error
state is set to the \textit{local error state}, and the exception is rethrown if
\texttt{(exceptions() \& badbit) != 0}}. \removed{(Exceptions thrown from
\texttt{basic_ios<>::clear()} are not caught or rethrown.) If
\texttt{(exceptions()\&badbit) != 0} then the exception is rethrown. It also
counts the number of characters extracted.} If no exception has been thrown it
\changed{ends by storing}{stores} the \changed{count}{number of characters
extracted} in a member object \removed{and returning the value specified}.
\added{After extraction is done, the input function calls \texttt{setstate},
which sets \texttt{*this}'s error state to the \textit{local error state}, and
may throw an exception.} In any event the sentry object is destroyed before
leaving the unformatted input function.
\end{quote}

In \textbf{[istream.unformatted] 27.7.4.3/4}:
\begin{quote}
\begin{codeblock}
int_type get();
\end{codeblock}
\textit{Effects:} Behaves as an unformatted input function (as described above).
After constructing a sentry object, extracts a character \texttt{c}, if one is
available. Otherwise, \changed{the function calls setstate(failbit), which may
throw \texttt{ios_base::failure} (27.5.5.4),}{\texttt{ios_base::failbit} is set
in the input function's \textit{local error state} before \texttt{setstate} is called.}
\end{quote}

In \textbf{[istream.unformatted] 27.7.4.3/6}:
\begin{quote}
\begin{codeblock}
basic_istream<charT, traits>& get(char_type& c);
\end{codeblock}
\textit{Effects:} Behaves as an unformatted input function (as described above).
After constructing a sentry object, extracts a character, if one is available,
and assigns it to \texttt{c}. Otherwise, \changed{the function calls
\texttt{setstate(failbit)} (which may throw \texttt{ios_base::failure} (27.5.5.4))}
{\texttt{ios_base::failbit} is set in the input function's \textit{local error state}
before \texttt{setstate} is called}.
\end{quote}

In \textbf{[istream.unformatted] 27.7.4.3/8}:
\begin{quote}
\begin{codeblock}
basic_istream<charT, traits>& get(char_type* s, streamsize n, char_type delim);
\end{codeblock}
\textit{Effects:} Behaves as an unformatted input function (as described above).
After constructing a sentry object, extracts characters and stores them into
successive locations of an array whose first element is designated by \texttt{s}.
Characters are extracted and stored until any of the following occurs:
\begin{itemize}
  \item[--] \texttt{n} is less than one or \texttt{n - 1} characters are stored;
  \item[--] end-of-file occurs on the input sequence\removed{ (in which case the
            function calls setstate(eofbit))};
  \item[--] \texttt{traits::eq(c, delim)} for the next available input character
            \texttt{c} (in which case \texttt{c} is not extracted).
\end{itemize}
If the function stores no characters, \changed{it calls \texttt{setstate(failbit)}
(which may throw \texttt{ios_base::failure} (27.5.5.4))}{\texttt{ios_base::failbit}
is set in the input function's \textit{local error state} before \texttt{setstate} is called}.
In any case, if \texttt{n} is greater than zero it then stores a
null character into the next successive location of the array.
\end{quote}

In \textbf{[istream.unformatted] 27.7.4.3/13}:
\begin{quote}
\begin{codeblock}
basic_istream<charT, traits>& get(basic_streambuf<char_type, traits>& sb, char_type delim);
\end{codeblock}
\textit{Effects:} Behaves as an unformatted input function (as described above).
After constructing a sentry object, extracts characters and inserts them in the
output sequence controlled by \texttt{sb}. Characters are extracted and inserted
until any of the following occurs:
\begin{itemize}
  \item[--] end-of-file occurs on the input sequence;
  \item[--] inserting in the output sequence fails (in which case the character
            to be inserted is not extracted);
  \item[--] \texttt{traits::eq(c, delim)} for the next available input character
            \texttt{c} (in which case \texttt{c} is not extracted);
  \item[--] an exception occurs (in which case, the exception is caught
            but not rethrown).
\end{itemize}
If the function inserts no characters, \changed{it calls \texttt{setstate(failbit)},
which may throw \texttt{ios_base::failure} (27.5.5.4)}{\texttt{ios_base::failbit}
is set in the input function's \textit{local error state} before \texttt{setstate} is called}.
\end{quote}

In \textbf{[istream.unformatted] 27.7.4.3/18}:
\begin{quote}
\begin{codeblock}
basic_istream<charT, traits>& getline(char_type* s, streamsize n, char_type delim);
\end{codeblock}
\textit{Effects:} Behaves as an unformatted input function (as described above).
After constructing a sentry object, extracts characters and stores them into
successive locations of an array whose first element is designated by \texttt{s}.
Characters are extracted and stored until one of the following occurs:
\begin{enumerate}
  \item end-of-file occurs on the input sequence\removed{ (in which case the
        function calls \texttt{setstate(eofbit)})};
  \item \texttt{traits::eq(c, delim)} for the next available input character
        \texttt{c} (in which case the input character is extracted but not stored);
  \item \texttt{n} is less than one or \texttt{n - 1} characters are stored
        (in which case the function calls \texttt{setstate(failbit)}).
\end{enumerate}

These conditions are tested in the order shown.

If the function extracts no characters, \changed{it calls \texttt{setstate(failbit)}
(which may throw \texttt{ios_base::failure} (27.5.5.4))}{\texttt{ios_base::failbit}
is set in the input function's \textit{local error state} before \texttt{setstate} is called}.

In any case, if \texttt{n} is greater than zero, it then stores a null character
(using \texttt{charT()}) into the next successive location of the array.
\end{quote}

In \textbf{[istream.extractors] 27.7.4.2.3/14}:
\begin{quote}
\begin{codeblock}
basic_istream<charT, traits>& operator>>(basic_streambuf<charT, traits>* sb);
\end{codeblock}
\textit{Effects}: Behaves as an unformatted input function (27.7.4.3). If \texttt{sb}
is null, calls \texttt{setstate(failbit)}, which may throw \texttt{ios_base::failure}
(27.5.5.4). After a sentry object is constructed, extracts characters from \texttt{*this}
and inserts them in the output sequence controlled by \texttt{sb}. Characters are
extracted and inserted until any of the following occurs:
\begin{itemize}
  \item[--] end-of-file occurs on the input sequence;
  \item[--] inserting in the output sequence fails (in which case the character to
            be inserted is not extracted);
  \item[--] an exception occurs (in which case the exception is caught).
\end{itemize}
If the function inserts no characters, \changed{it calls \texttt{setstate(failbit)},
which may throw \texttt{ios_base::failure} (27.5.5.4)}{\texttt{ios_base::failbit}
is set in the input function's \textit{local error state} before \texttt{setstate} is called}.
\removed{If it inserted no characters because it caught an exception
thrown while extracting characters from \texttt{*this} and \texttt{failbit} is on
in \texttt{exceptions()} (27.5.5.4), then the caught exception is rethrown.}
\end{quote}

In \textbf{[string.io] 20.3.3.9/6}:
\begin{quote}
\begin{codeblock}
template<class charT, class traits, class Allocator>
basic_istream<charT, traits>&
  getline(basic_istream<charT, traits>& is,
          basic_string<charT, traits, Allocator>& str,
          charT delim);
template<class charT, class traits, class Allocator>
basic_istream<charT, traits>&
  getline(basic_istream<charT, traits>&& is,
          basic_string<charT, traits, Allocator>& str,
          charT delim);
\end{codeblock}
\textit{Effects:} Behaves as an unformatted input function (27.7.4.3), except
that it does not affect the value returned by subsequent calls to \texttt{basic_istream<>::gcount()}.
After constructing a sentry object, if the sentry converts to \texttt{true},
calls \texttt{str.erase()} and then extracts characters from \texttt{is} and
appends them to \texttt{str} as if by calling \texttt{str.append(1, c)} until
any of the following occurs:
\begin{itemize}
  \item[--] end-of-file occurs on the input sequence\added{;}\removed{(in which case, the
            \texttt{getline} function calls \texttt{is.setstate(ios_base::eofbit)}).}
  \item[--] \texttt{traits::eq(c, delim)} for the next available input character
            \texttt{c} (in which case, \texttt{c} is extracted but not appended)\changed{ (27.5.5.4)}{;}
  \item[--] \texttt{str.max_size()} characters are stored (in which case,
            \changed{the function calls \texttt{is.setstate(ios_base::failbit)} (27.5.5.4)}
            {\texttt{ios_base::failbit} is turned on in the input function's \textit{local error state}})
\end{itemize}
The conditions are tested in the order shown. In any case, after the last
character is extracted, the sentry object is destroyed.

If the function extracts no characters, \changed{it calls
\texttt{is.setstate(ios_base::failbit)} which may throw
\texttt{ios_base::failure} (27.5.5.4)}{\texttt{ios_base::failbit} is turned
on in the input function’s \textit{local error state} before the stream’s error state
is updated}.
\end{quote}


\section{Appendix: a few test cases}
This section contains test cases that were handled in different ways by the
implementations. They are provided as a proof that we need to solve the problem,
and for the implementer's reference if they deem it useful.

First, let's introduce a few definitions from the Standard so we can refer to
them below.

\begin{enumerate}
  \item[(A)] \textbf{[istream] 27.7.4.1/3} (applies to both formatted and unformatted input functions):
  \begin{quote}
  If \cc{rdbuf()->sbumpc()} or \cc{rdbuf()->sgetc()} returns \cc{traits::eof()},
  then the input function, except as explicitly noted otherwise, completes its
  actions and does \cc{setstate(eofbit)}, which may throw \cc{ios_base::failure}
  (27.5.5.4), before returning.
  \end{quote}

  \item[(B)] \textbf{[istream] 27.7.4.1/4} (applies to both formatted and unformatted input functions):
  \begin{quote}
  If one of these called functions throws an exception, then unless explicitly
  noted otherwise, the input function sets \cc{badbit} in error state. If
  \cc{badbit} is on in \cc{exceptions()}, the input function rethrows the
  exception without completing its actions, otherwise it does not throw
  anything and proceeds as if the called function had returned a failure
  indication.
  \end{quote}

  \item[(C)] \textbf{[istream.formatted.reqmts] 27.7.4.2.1/1} (applies only to formatted input functions):
  \begin{quote}
  Each formatted input function begins execution by constructing an object of
  class \cc{sentry} with the \cc{noskipws} (second) argument \cc{false}. If
  the sentry object returns \cc{true}, when converted to a value of type
  \cc{bool}, the function endeavors to obtain the requested input. If an exception
  is thrown during input then \cc{ios::badbit} is turned on in \cc{*this}'s error
  state. If \cc{(exceptions()&badbit) != 0} then the exception is rethrown. In
  any case, the formatted input function destroys the sentry object. If no
  exception has been thrown, it returns \cc{*this}.
  \end{quote}

  \item[(D)] \textbf{[istream.unformatted] 27.7.4.3/1} (applies only to unformatted input functions):
  \begin{quote}
  [...] If an exception is thrown during input then \cc{ios::badbit} is turned
  on in \cc{*this}'s error state. (Exceptions thrown from \cc{basic_ios<>::clear()}
  are not caught or rethrown.) If \cc{(exceptions()&badbit) != 0} then the
  exception is rethrown. It also counts the number of characters extracted.
  If no exception has been thrown it ends by storing the count in a member
  object and returning the value specified. In any event the sentry object
  is destroyed before leaving the unformatted input function.
  \end{quote}
\end{enumerate}

With all this laid out, here's a couple of test cases:

\begin{enumerate}
  \item Formatted input operation which fails to extract from a non-empty stream
  \begin{cpp}
  #include <iostream>
  #include <sstream>
  int main () {
      std::stringbuf buf("not empty");
      std::istream is(&buf);
      is.exceptions(std::ios::failbit);

      bool threw = false;
      try {
          unsigned int tmp{};
          is >> tmp;
      } catch (std::ios::failure const&) {
          threw = true;
      }

      std::cout << "bad = " << is.bad() << std::endl;
      std::cout << "fail = " << is.fail() << std::endl;
      std::cout << "eof = " << is.eof() << std::endl;
      std::cout << "threw = " << threw << std::endl;
  }
  \end{cpp}

  The current behavior is the following:
  \begin{center}
  \begin{tabular}{| l | l | l | l |}
  \hline
         & \href{https://wandbox.org/permlink/fVgU3C1cZWXwbhAN}{libstdc++}
         & \href{http://rextester.com/CBDQE38523}{MSVC}
         & \href{https://wandbox.org/permlink/7sHlkXB3hBqZZ1Ge}{libc++} \\ \hline
  bad    & 0         & 0    & 1      \\ \hline
  fail   & 1         & 1    & 1      \\ \hline
  eof    & 0         & 0    & 0      \\ \hline
  threw  & 1         & 1    & 0      \\ \hline
  \end{tabular}
  \end{center}

  My interpretation is that per the definition of \cc{operator>>(unsigned int&)} in \textbf{[istream.formatted.arithmetic] 27.7.4.2.2/1}, we try to extract an \cc{unsigned int} from the stream:
  \begin{cpp}
    using numget = num_get<charT, istreambuf_iterator<charT, traits>>;
    iostate err = iostate::goodbit;
    use_facet<numget>(loc).get(*this, 0, *this, err, val);
    setstate(err);
  \end{cpp}
  This \cc{num_get::get} fails because the format is wrong and reports that by setting \cc{err} to \cc{std::ios_base::failbit}, which results in \cc{setstate(err)} throwing because \cc{failbit} had been set in the exceptions. I don't think (B) applies here because the exception is not being thrown as part of \cc{rdbuf()->sbumpc()} or \cc{rdbuf()->sgetc()}. However, (C) seems to apply, which means that we catch the exception and set badbit on the stream, but we do not rethrow the exception because \cc{badbit} is not set in \cc{exceptions()}. Hence, libc++'s behavior seems correct to me, despite being useless.

  \item Formatted input operation which fails to extract from an empty stream
  \begin{cpp}
  #include <iostream>
  #include <sstream>
  int main () {
      std::stringbuf buf; // empty
      std::istream is(&buf);
      is.exceptions(std::ios::failbit);

      bool threw = false;
      try {
          unsigned int tmp{};
          is >> tmp;
      } catch (std::ios::failure const&) {
          threw = true;
      }

      std::cout << "bad = " << is.bad() << std::endl;
      std::cout << "fail = " << is.fail() << std::endl;
      std::cout << "eof = " << is.eof() << std::endl;
      std::cout << "threw = " << threw << std::endl;
  }
  \end{cpp}

  The current behavior is the following:
  \begin{center}
  \begin{tabular}{| l | l | l | l |}
  \hline
         & \href{https://wandbox.org/permlink/UpOSzH76Ovm4RzTz}{libstdc++}
         & \href{http://rextester.com/WBRK78783}{MSVC}
         & \href{https://wandbox.org/permlink/aL4Xl2d8VKVK2EY2}{libc++} \\ \hline
  bad    & 0         & 0    & 1      \\ \hline
  fail   & 1         & 1    & 1      \\ \hline
  eof    & 1         & 1    & 1      \\ \hline
  threw  & 1         & 1    & 0      \\ \hline
  \end{tabular}
  \end{center}

  My interpretation is that per (C), we create a sentry object which attempts to skip whitespace and fails because we're at the end of file. The sentry calls \cc{setstate(failbit | eofbit)}, which throws an exception because failbit is set in the exceptions. We then set \cc{badbit} on the stream and do not rethrow the exception, because \cc{badbit} is not in the exceptions. Also note that I don't think (B) applies here, because we never make it to the operations specified in (A), which I think is what (B) is referring to. Hence, libc++ is correct again here.

  \item Unformatted input operation which hits EOF
  \begin{cpp}
  #include <iostream>
  #include <sstream>
  int main() {
      std::stringbuf sb("rrrrrrrrr");
      std::istream is(&sb);
      is >> std::noskipws;
      is.exceptions(std::ios::eofbit);

      bool threw = false;
      try {
          while (true) {
              is.get();
              if (is.eof())
                  break;
          }
      } catch (std::ios::failure const&) {
          threw = true;
      }

      std::cout << "bad = " << is.bad() << std::endl;
      std::cout << "fail = " << is.fail() << std::endl;
      std::cout << "eof = " << is.eof() << std::endl;
      std::cout << "threw = " << threw << std::endl;
  }
  \end{cpp}

  The current behavior is the following:
  \begin{center}
  \begin{tabular}{| l | l | l | l |}
  \hline
         & \href{https://wandbox.org/permlink/jSSGM6TcqLzZSl6M}{libstdc++}
         & \href{http://rextester.com/OPIJW60076}{MSVC}
         & \href{https://wandbox.org/permlink/O302uzC1VW0nW1Pn}{libc++} \\ \hline
  bad    & 0         & 0    & 1      \\ \hline
  fail   & 1         & 1    & 1      \\ \hline
  eof    & 1         & 1    & 1      \\ \hline
  threw  & 1         & 1    & 0      \\ \hline
  \end{tabular}
  \end{center}

  My interpretation is that we create the sentry, which doesn't do much because we're not trying to skip whitespace. We then try to extract a character and fail because we hit the end of file. Per the definition of \cc{basic_istream::get()} in \textbf{[istream.unformatted] 27.7.4.3/4}, we call \cc{setstate(failbit)}, which throws an exception. Per (A), we're also somehow required to call \cc{setstate(eofbit)}. Finally, per (D), we also set \cc{badbit} on the stream, and we don't rethrow any exception because \cc{badbit} is not in the exceptions. I think this makes libc++ right again.

  Actually, I don't think this specification can be implemented as-is because of \cite{LWG61}, which added the part "(Exceptions thrown from \cc{basic_ios<>::clear()} are not caught or rethrown.)". This would make it effectively impossible to call both \cc{setstate(failbit)} and \cc{setstate(eofbit)}, and also to set the \cc{badbit} on the stream. Unless I'm missing a clever implementation trick, you basically have to catch and rethrow.
\end{enumerate}


\section{References}
\renewcommand{\section}[2]{}%
\begin{thebibliography}{9}

\bibitem[N4778]{N4778}
  Richard Smith,
  \emph{Working Draft, Standard for Programming Language C++}\newline
  \url{http://www.open-std.org/jtc1/sc22/wg21/docs/papers/2018/n4778.pdf}

\bibitem[LWG2349]{LWG2349}
  Zhihao Yuan,
  \emph{Clarify input/output function rethrow behavior}\newline
  \url{https://cplusplus.github.io/LWG/issue2349}

\bibitem[LWG61]{LWG61}
  Matt Austern,
  \emph{Ambiguity in iostreams exception policy}\newline
  \url{https://cplusplus.github.io/LWG/issue61}

\end{thebibliography}

\end{document}
